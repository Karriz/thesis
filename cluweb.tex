% Re-engineered ClusterWeb Chapter

The re-engineering project of Cluster Web has the aim to make the tool more user-friendly and also expandable to other purposes than Cluster II pass scheduling. In this section the user requirements, development process and the technological architecture of the new Cluster Web (named "CluWeb") is overviewed.

%Pictures in .eps if you use latex, .pdf or .png if you use pdflatex. Don't specify the extension so you can use both!
\begin{figure}[ht]
  \begin{center}
    \includegraphics*[width=1\textwidth]{cluweb_dev}
  \end{center}
  \caption{A screenshot of the in-development CluWeb}
  \label{fig:cluweb}
\end{figure}

\section{Requirements}
    The main requirements for the re-engineered Cluster Web are related about making it easier and faster to use and also more configurable. There is also interest from other missions such as ExoMars and Swarm to use a similar tool for planning and overviewing operations. This means that the technical implementation has to be made modular so that it can be expanded with different kinds of visualizations and functionality.

One of the problems with the old Cluster Web is the slowness of navigating the timeline. It can only be moved and zoomed in steps by clicking the buttons in the navigation bar or by selecting a date in the calendar. Each time the timeline moves it is re-drawn and that can take a few seconds. It also takes too many clicks to create passes.

[CluWeb Pass timeline representation brainstorming]

\section{Development}
This section will detail different tools and methods used in CluWeb's development process to manage the development and to streamline and automatize certain phases like testing and deployment.

\subsection{Scrum}
CluWeb development uses the Scrum approach, which is an Agile framework. [http://www.scrumguides.org/docs/scrumguide/v2016/2016-Scrum-Guide-US.pdf]
In Scrum, the development process is split into "sprints" which in the case of CluWeb happen every two weeks. Sprints contain "user stories" defined by the project owner which describe some functionality from a user's point of view. A sprint planning meeting is held at a beginning of a sprint to choose the user stories in that sprint and to score them by their workload in three different categories: concept, front-end and back-end. User stories are split into technical tasks which are then assigned among the developers. A Scrum master who is a member of the development team manages these meetings.

During a sprint, a daily stand-up meeting is held every morning during which the developers tell others what they did during the last day and what they will be doing that day. This is to stay updated of everyone's progress and to exchange ideas and ask for help if necessary.

Task status is tracked on a Kanban board. There are five different stages for tasks: "new", "in progress", "ready for testing", "closed" and "needs info". Once a developer has completed a task, it is moved to "ready for testing" stage, after which another development tests the functionality and closes the task. Once all tasks in a user story have been completed, it can be closed.

At the end of a sprint, a sprint review meeting is held in which the progress is presented to the project owner who provides feedback. Then another sprint can start with a new set of user stories, and unfinished user stories. which were left over from the last sprint.

Taiga project management platform is used for managing the Scrum process.

\subsection{Continuous integration and deployment}
CluWeb's code is managed in two GitLab repositories, one for front-end and another for back-end. Initially both used to be in one repository but they were separated for easier and cleaner management.

Both repositories have a master branch that is supposed to contain stable release code, and a development branch that contains new features which have been added after every sprint. User stories are developed separately in their own branches which are based on development branch, and merged back into development branch when ready.

GitLab has built-in continuous integration features which are utilized in CluWeb's repositories. It's possible to define scripts defining different jobs which are run automatically after a push on a GitLab runner that is a Docker container running on the GitLab server. The idea is to run test first and then deploy the new version on a server.

GitLab's continuous integration jobs are separated into stages which are run in a sequence. In CluWeb's case what needs to be achieved is to run tests first to validate the commits which were pushed, and if the tests were successful, deploy the new version to a server. Specific jobs can be set to run only for specific branches, which would for example allow deployment to branch-specific servers.

Front-end tests are written using Jasmine framework, and run using Karma test runner on PhantomJS headless browser. The tests define expected outcomes for certain function calls, run them and then report the results.

Something about backend testing

Deployment is done after the test stage has passed. For the front-end this is done by connecting to the server using SSH, pulling the commit there, and then compiling the Typescript. Then the compiled Javascript is copied into a static files folders served by Nginx.

Back-end deployment similarly connects to the server with SSH, pulls the commit and then runs a script to migrate any model changes to the database. Django automatically restarts the server when changes are detected.

Figure There

\section{Architecture}
CluWeb uses modern web technologies which are in widespread use across the industry. The overall architecture consists of a front- and a back-end which are separate software components. Front-end provides a browser-based user interface, while the back-end together with a database serves data to the front-end. As long as the interfaces stay the same, the individual components could be updated or replaced.

%Pictures in .eps if you use latex, .pdf or .png if you use pdflatex. Don't specify the extension so you can use both!
\begin{figure}[ht]
  \begin{center}
    \includegraphics*[width=1\textwidth]{architecture}
  \end{center}
  \caption{Overall software architecture of CluWeb}
  \label{fig:cluweb_architecture}
\end{figure}

\subsection{Front-end}
CluWeb's front-end is powered by Angular, which is a Javascript framework for developing web applications. It provides useful functionality for structuring the application logic and different views and services.

An Angular application consists of modules which consist of different components and services. Components contain the application logic and have different lifecycle events that fire when the view is created or destroyed, for example.

Each component has a template that defines the user interface of the component in HTML. The template can have data bindings to the component's variables inside HTML, reflecting changes in data. Bindings also work the other way around, and can be used for inserting user input into the component's data.

Services provide data that components can obtain through change listeners or by calling request functions. They are injected in the component constructor.

Angular applications are programmed in Typescript which is a superset of Javascript that adds optional static typing, making the code easier to read and maintain than regular Javascript.

For drawing the schedule and visualizing data on it, D3.js is used. It is a Javascript library for binding data to DOM elements, and applying transformations to the elements based on that data. In CluWeb's case these features are used for drawing a timeline axis and placing elements on the timeline based on the corresponding data objects' properties like start and end time and spacecraft number.

\subsection{Back-end}
Back-end of CluWeb is built using the Django Python framework. The back-end defines data models, interfaces with a database and provides the data through a RESTful API in JSON format.

The back-end also has importers for various different file formats so that new data can be added to the database. Some of the data is retrieved from external databases. Passes, ground station bookings and visibilities come from ESTRACK's API.

SSR fill level is retrieved from ESA's telemetry-collecting MUSTLink database when requested and processed to remove invalid and unnecessary data. Fill level is a derived value and not a telemetry value from the spacecraft itself. It is calculated by dividing the difference between read and write pointers with the total storage capacity. The value can only be properly calculated during a pass when SSR is in standby or reproduce mode, i.e. reading data and downlinking it. 

Between passes there is no telemetry, so those times can be ignored as the SSR fill level would just show as zero.

For passes it's not optimal to get values for every frame (about 15 seconds) as that would be too much data for drawing the SSR fill level for long time periods. What needs to be retrieved is the times when the slope of the SSR fill level changes, and just draw the line between those points of time. Getting the points where SSR changes to either standby or reproduce mode or back to record is enough for achieving this.

\section{Usability}
CluWeb has the aim of providing a faster and more intuitive user experience. One of the main differences to the old Cluster Web is that the user can drag and zoom the schedule in real time by using a mouse. There's no need to redraw the entire schedule when zooming or dragging, instead it just translates to the new position at the same rate with the mouse cursor. On a touch screen the schedule can be dragged with a finger and zoomed by pinching.

User can also choose a range between two dates on a calendar, and then the schedule is moved to that range through an animation. This is intended to give the user a better idea of where in time the schedule moved, rather than just directly jumping to a new time.

 A new feature that has been added is the "live mode" in which the schedule shows a default timerange in past and future and moves with the current time. This could potentially be useful for getting a real-time overview of ongoing activities in the mission control room. The time range can be configured, so in more long-term operations it could be a few days, while during more short-term mission events it could be just minutes.
 
Screenshot here

\section{Configurability}
Making CluWeb configurable for different purposes was one of the driving factors for the re-engineering project. Ultimately, users should be able to visualize anything they want on the schedule, not necessarily even in the context of spacecraft operations. To achieve this, the schedule configuration needs to be defined in a way that can be easily edited, instead of having the attributes of the schedule and the visualizations hard-coded.

In Cluster Web, the main view is a schedule that consists of four timelines stacked vertically, one for each spacecraft. The timelines then visualize different groups of data on top of each other. The definitions for the timelines and the visualizations are hard-coded which means that they cannot be quickly changed by the user.

Re-engineered CluWeb fixes this issue by introducing a JSON-based configuration grammar that can be used to define every aspect of the schedule. User can add any amount of timelines on the schedule and define their sizes, what data is visualized on them and and in what order the visualizations are shown. Attributes of the visualizations themselves can also be edited, like the color of the elements.

\subsection{Visualization types}  \label{vis_types}
Finding similarities between features was an important step of making CluWeb more generic than its predecessor. Because a lot of the visualizations in Cluster Web had common attributes between them it made sense to make superclasses out of those commonalities from which the different visualizations could then inherit. The following visualization categories were decided on:

\begin{itemize}
\item Time intervals (passes, visibilities, bookings, payload sequences)
\item Time instants (maneuvers, apogees/perigees)
\item States (TDA modes)
\item Time series (SSR fill level)
\end{itemize}

Time intervals are any data in which an instance has a defined beginning and end time. They're usually visualized by a rectangle that has starting time on its left edge on the horizontal axis and ending time on the right edge.

Time instants are data that contain only one point in time per instance, i.e. events which don't have a defined length. These are visualized by a point in time.

State data contains times in which some state changes. There can only be one state at a time, so whenever a state changes it replaces the previous one, and lasts until the next change. These can be visualized by the timeline background color.

Time series data contains numerical values at different points in time. The possible visualizations for this type of data can be for example line or bar charts.

In code these four categories contain the logic for what base attributes the data has and how it is visualized as an element on the timeline. Specific attributes such as color and height of the elements can then be edited without needlessly repeating  code.

\subsection{Schedule configuration grammar}
In the configuration file, the top level object is the schedule itself. It defines certain attributes like width and height, maximum and minimum zoom level and time margins to be displayed in the live mode.

Schedule can contain any number of timelines. Each timeline has it's own attributes, most importantly the data domain it displays. This can be for example be the id of a spacecraft or a ground station. Data that references to that particular domain is displayed on that timeline. 

Each timeline also has a weight for defining the relative heights of the timelines, and a label which are displayed in the top left corners of the timelines.

Timelines contain a list of what data visualizations are displayed on them. The data visualizations define the data source, data type (as explained in \ref{vis_types}), display z-index of the visualization and the domain by which the data should be placed on different timelines. For example if the data type "pass" defines a domain "spacecraft", and a timeline containing that data type has the domain "CLU1", this means that the timeline will only display passes of Cluster 1. Alternatively, a timeline could for example show all passes of one specific ground station.

Other attributes of data visualizations vary depending on the data type. For time intervals these are for example color, height and label text.

%Pictures in .eps if you use latex, .pdf or .png if you use pdflatex. Don't specify the extension so you can use both!
\begin{figure}[ht]
  \begin{center}
    \includegraphics*[width=1\textwidth]{schedule_diagram}
  \end{center}
  \caption{A diagram describing the relationship between timelines and data types in the schedule configuration}
  \label{fig:schedule_diagram}
\end{figure}


\section{Überlog integration}
Like the old Cluster Web, CluWeb integrates the viewing of Überlog entries, however in a different way. In CluWeb, the log entries are visible as icons on the timeline, their positions corresponding the log event time. 

Because there are so many entries close to each other, it would not be viable to display them all as the icons would overlap each other when zoomed out. To improve the readability of the timeline, log entries which are close are grouped together and shown as a single icon with a number next to it. The icon being displayed depends on the highest severity level log entry in that group. The groups change dynamically when zoom level changes. This solution was inspired by PowerHA timeline displays log entries at different severity levels in a similar way. [http://www.iftekhar.me/ibm/ibm-project-timeline]

When the user hovers the cursor over an icon, the times of the first and last entry in the group are displayed in a tooltip as well as the amount of entries for each severity level. If there is only one entry for the icon, the actual text of the log entry is displayed.

The icon grouping was implemented as a generic class that can be extended for other purposes as well.
 
%Pictures in .eps if you use latex, .pdf or .png if you use pdflatex. Don't specify the extension so you can use both!
\begin{figure}[ht]
  \begin{center}
    \includegraphics*[width=1\textwidth]{cluweb_uberlog}
  \end{center}
  \caption{A screenshot demonstrating how Uberlog entries are displayed on the timeline}
  \label{fig:cluweb_uberlog}
\end{figure}
