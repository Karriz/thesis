% Evaluation Chapter

This section covers the evaluation done on the old Cluster Web and the new CluWeb. The aim of the evaluation is to justify the need for a re-engineered Cluster Web and to gain qualitative and quantitative data about how the user experience was improved between the different versions.

Something human computer interaction \cite{4839639}

The evaluation was done in two phases: at first, focusing only on how, why and when operations engineers and spacecraft controllers use the old Cluster Web and what are its strengths and shortcomings. The results can then help in understanding what is required from the new CluWeb to fullfill the users' needs and on what areas in particular should be a focus for improving the user experience.

In the second phase of the evaluation, a comparative evaluation between Cluster Web and CluWeb was performed to find out how it improved in terms of user experience. CluWeb wasn't functionally yet on par with the old Culster Web at this point of development as it was missing the pass planning functionality entirely, so the evaluation focused on two aspects which could be compared: monitoring and data visualization.

\section{Definitions}

There are many different established methods for evaluating software usability and user experience. At first it is important to define what exactly these terms mean so that there is no misunderstanding.

The definitions of usability and user experience in the context of software evaluation aren't self-evident, and they can be interpreted in many different ways. However, these terms do have official ISO standards defining them. Bevan et al.talk about the importance of using these standards so that the criteria against which software is evaluated stays consistent. \cite{bevanstandard}

In  ISO FDIS 9241-210, usability is defined as "Extent to which  a system, product or service can be used by specified users to achieve specified goals with effectiveness, efficiency and satisfaction in a specified context of use." Therefore, usability seems to have its main focus on the pragmatic goal of achieving some task in a efficient way while using the system, however it also includes the concept of satisfaction.

User experience on the other hand is defined as "A person's perceptions and responses that result from the use and/or anticipated use of a product, system or service." It would appear that user experience is more about deriving pleasure from using a system even if it is not for entertainment purposes.

There are many ways to interpret how these terms relate to each other. If the concept of "satisfaction" in the definition of usability is considered to cover "a person's perceptions and responses" in the definition of user experience, then usability can be seen as a umbrella term that covers both the actual measurable work efficiency and personal feelings that stem from using the system.

Bevan notes that in industry it is often user experience that is used as an umbrella term that includes the work efficiency component of usability, while in research community user experience is seen as the subjective perception that user gets from using a system.

In the end there are three different interpretations of user experience according to Bevan: \begin{itemize}
\item An  elaboration  of  the  satisfaction  component  of usability
\item Distinct  from  usability,  which  has  a  historical emphasis on user performance
\item An  umbrella  term  for  all  the  user’s  perceptions  and responses,  whether  measured  subjectively  or objectively
\end{itemize}
\cite{bevan2009difference}

Petrie and Bevan further open up the meanings of these terms and their components and how they're used in research community. Usability can be contain concepts like learnability, flexibility, memorability, safety and accessibility in addition to efficiency, because all these contribute to achieving some end goal through the use of the system. What exactly is considered usability is dependent to the system in question. 

According to Petrie and Bevan, users may want more from their interaction with the system than just to complete a task efficiently, and this is where user experience is an important thing to consider. Information technology has become an ubiquitous part of everyday life and is no longer just a means to an end to people.
\cite{bevanevaluation}

\cite{bevan2009difference, bevaniso, bevanevaluation, bevanstandard}

Tullis and Albert consider user experience to be a broad term that covers the user's entire interaction with the system, including their thoughts, feelings and perceptions. Their view of user experience metrics is that they reveal information about the effectiveness, efficiency and satisfaction that's a result of the interaction between the user and the system. \cite{albert2013measuring}

Hassenzahl et al. look at usability and user experience from different points of view and argue that while these terms practically cover the same things, they have slightly different focuses. Usability is more concerned with practical task completion and objectively measurable metrics, while user experience tries to balance the practical and hedonic sides of information system usage and also considers how the user feels after using the system.
\cite{hassenzahl2006user}

As it can be seen, there is no clear consensus on whether user experience is a part of usability or vice versa, or whether they are two separate concepts. There is support for many different interpretations. 

Because data visualization and monitoring is the part of CluWeb that is going to be evaluated, this puts less focus on clear-cut task completion. Pass planning would for example be a well-defined task that is very well suited for purely pragmatic usability evaluation, but it is absent from the current iteration of CluWeb. Inspecting a data visualization is by nature a more organic process for the user.

A schedule monitoring screen is continuously presenting information in the flight control room. It could be argued that it is a "public display", albeit there is only a small number of users in its vicinity, as opposed to a public display that is for example showing timetables at a bus stop.

For consistency's sake to not use words interchargeably in this thesis, "user experience" is going to be used as an all-encompassing term that covers every aspect of the user's interaction with the system, from their ability to complete tasks to what kind of feelings stem from the usage of the system. Both pragmatic and hedonic goals of information system usage should be considered, and user experience seems to be a better word for this than just usability. 

\section{Methods}

\section{Results}