Spacecraft operations require a lot of software for planning, operating and analyzing purposes. Mission-critical software that is for example used for controlling spacecraft is often based on decades-old technologies and is subject to very strict requirements, however for non-critical planning and analyzing purposes software can be developed in a much more agile way using modern tools, which allows experimenting with different technologies and ideas quickly with the goal of finding ways to improve the workflow of spacecraft operators with new software.

European Space Agency's Cluster II flight control team utilizes a PHP-based tool called "Cluster Web" for planning ground station passes for each of the four spacecraft in the Cluster II constellation. It is a timeline interface that can visualize different kinds of mission data that is relevant to the mission planners, and allows them to edit and export a schedule that is then used for the actual mission planning system.

While Cluster Web serves its purpose for Cluster II mission planning, there are some usability issues and areas which could be improved. Other missions also have interest in a similar tool. Because Cluster Web is built on aging technologies and not very easily expandable in its architecture, a re-engineering project called "OpsWeb" was started with the aim of creating a generic timeline visualization tool with modern web technologies like Angular 2 and Django.

OpsWeb's architecture is designed from scratch to be as configurable as possible for any potential use case. It is also designed to provide a more fluid user experience.

This thesis aims to answer several questions regarding Cluster Web and OpsWeb. First, in chapter \ref{usability_chapter} there is some background research regarding what are usability and user experience, and how they can be measured. Then, in chapter \ref{operations_chapter}, the Cluster II mission is introduced, in particular how the mission planning works and what is the purpose of Cluster Web. Different features and the technical architecture of Cluster Web are overviewed.

Chapter \ref{cluweb_chapter} is intended to answer why OpsWeb is needed, what features it has, how it works and how it is being developed. Then in chapter \ref{evaluation_chapter} the evaluation of Cluster Web and OpsWeb is described and the results are overviewed. The purpose of the evaluation is to find out if and how certain things like efficiency and perceived user experience changed between Cluster Web and OpsWeb. Finally, in chapter \ref{discussion_chapter} the overall results of the OpsWeb project and the evaluation are overviewed.