Spacecraft operations require a lot of software for planning, operating and analyzing purposes. Mission-critical software that is for example used for controlling spacecraft is often based on decades-old technologies and is subject to very strict requirements, however for non-critical planning and analyzing purposes software can be developed in a much more agile way using modern tools, which allows experimenting with different technologies and ideas quickly with the goal of finding ways to improve the workflow of spacecraft operators with new software.

European Space Agency's Cluster II flight control team utilizes a PHP-based tool called "Cluster Web" for plannign ground station passes for each of the four spacecraft in the Cluster II constellation. It is a timeline interface that can visualize different kinds of mission data that is relevant to the mission planners, and allows them to edit and export a schedule that is then used for the actual mission planning system.

While Cluster Web serves its purpose for Cluster II mission planning, there is interest for a similar tool at other missions, and for adding new features which would improve usability. Because Cluster Web is built on aging technologies and not very easily expandable in its architecture, a re-engineering project called "CluWeb" was started with the aim of creating a generic timeline visualization tool with modern web technologies like Angular 2 and Django.

This thesis explains what the purpose of Cluster Web is in the context of spacecraft operations, provides an overview to the development process, new features and architecture of CluWeb and does a comparative study between it and Cluster Web to find out how usability was improved, and how CluWeb makes future development streamlined through a configurable and modular architecture.