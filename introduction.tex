The field of spacecraft operations requires a lot of software for planning, operating and analyzing space missions. Mission-critical software that is for example used for controlling spacecraft is often based on decades-old technologies and is subject to very strict requirements, however for non-critical planning and analyzing purposes software can be developed in a much more agile way using modern tools.

European Space Agency's Cluster II flight control team utilizes a tool called "Cluster Web" for plannign ground station passes for each of the four spacecraft in the Cluster II constellation. It is a timeline interface that can visualize different kinds of mission data that is relevant to the mission planners. \cite{kamara, al-shaer}


\section{Sample section with a table reference}

Ut hendrerit volutpat felis vitae aliquam. Duis quis augue urna. In sollicitudin lacinia elit, 
non ultrices dui tristique eu. In hac habitasse platea dictumst. Nullam mi sapien, sagittis non 
mi in, gravida lobortis ante. A sample latex table can be seen in Table~\ref{tab:sample_table}.


\begin{table}[!ht]
% Add some padding to the table cells:
\def\arraystretch{1.1}%
\begin{center}
  \caption{Sample table}
  \label{tab:sample_table}
  \begin{tabular}{| l | c | }
    \hline
    Sample & table \\
    \hline
    Sample & table \\
    Sample & table \\
    \hline
  \end{tabular}

  \end{center}
\end{table}

\section{Changelog}

\begin{itemize}
\item Changed \textit{\textbackslash{chapter's}} \textbackslash{newpage} to 
\textbackslash{clearpage} to prevent floats from wandering to the beginning of the next chapter

\item Added \textbf{[hyphens]} to the url package to prevent margin overflow with 
long urls

\item Added \textbf{multirow} package to make multirow and multicolumns possible

\item Added some helpful source code comments

\item Makefile for pdflatex and bibtex to automate pdf compilation

\item Abbreviations are autosorted by the Makefile

\item Added a bit of extra padding to the sample table
\end{itemize}

\subsection{Sample subsection with a figure reference}

Sed erat neque, cursus ac feugiat ac, sollicitudin
ut odio. Maecenas vel turpis rhoncus, euismod nisl ac, tincidunt ipsum. Curabitur fermentum vel
turpis ac lobortis. Cras a justo vitae diam volutpat blandit. Maecenas faucibus nibh a neque 
semper ullamcorper. Suspendisse in est vulputate, fermentum odio nec, pharetra augue. Fusce at
consequat arcu, sed hendrerit enim. Pellentesque id suscipit nibh, id pretium erat. 

Nam eget libero neque. Nullam commodo cursus turpis mollis cursus. Curabitur est tellus,
pellentesque eu velit sed, ullamcorper gravida felis. Proin vel cursus risus, at scelerisque 
justo. Quisque rutrum justo at ultricies auctor. A sample latex figure can be seen in
Figure~\ref{fig:oylogoe}. If your pictures appear grainy, you probably have too low dots
per inch (DPI) value.

\footnotetext[1]{Sample footnote}

%Pictures in .eps if you use latex, .pdf or .png if you use pdflatex. Don't specify the extension so you can use both!
\begin{figure}[ht]
  \begin{center}
    \includegraphics*[width=0.3\textwidth]{oylogoe}
  \end{center}
  \caption{A picture}
  \label{fig:oylogoe}
\end{figure}

