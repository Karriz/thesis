%
% Template for Department of Electrical and Information Engineering Diploma Thesis v1.1.2013
% Authors: Mika Korhonen (original author), Pekka Pietikäinen, Christian Wieser, Teemu Tokola and Juha Kylmänen.
% If you make any improvements to this template, please contact ouspg@ee.oulu.fi
%

\documentclass[a4paper, 12pt,titlepage]{dithesis}
\usepackage[english,finnish]{babel}
\usepackage[utf8]{inputenc}
\usepackage[T1]{fontenc}  
\usepackage{times}
\usepackage{tabularx}
\usepackage{graphicx}
\usepackage{float}
\usepackage{enumerate}
\usepackage{placeins}
\usepackage{fancybox}
\usepackage{verbatim}
\usepackage{longtable}
\usepackage{di}
\usepackage[hyphens]{url}
\usepackage{boxedminipage}
\usepackage{subfigure}
\usepackage{multirow}
\usepackage{amsfonts}
\usepackage{xcolor}
\tolerance=500

%\usepackage[a4paper,margin=2.5cm,dvips]{geometry}
%\geometry{papersize={210mm,297mm}}
%dvipdf -sPAPERSIZE=a4

% The following code removes %-signs with URL:s longer than 72 chars
\begingroup
\makeatletter
\g@addto@macro{\UrlSpecials}{%
  \endlinechar=13 \catcode\endlinechar=12
  \do\%{\Url@percent}\do\^^M{\break}}
 \catcode13=12 %
 \gdef\Url@percent{\@ifnextchar^^M{\@gobble}{\mathbin{\mathchar`\%}}}%
\endgroup %

%\selectlanguage{finnish}

\otsikko{Satelliittioperaattoreiden työnkulun optimointi uusien front-end teknologioiden avulla}
\title{Optimization of spacecraft operator workflow through state-of-the-art front-end technologies}

\etunimi{Karri}
\sukunimi{Ojala}
\valvoja{}
\koulutusohjelma{information} % {information | electrical}
\vuosi{2017}
\tyo{Master} % {Bachelor | Master}
\kieli{english} % {finnish | english}

\begin{document}

\begin{titlepage}
	\centering{\includegraphics*[width=0.3\textwidth]{uni_logo}\\}
	{\sffamily\fontsize{9}{1pt}\selectfont FACULTY OF INFORMATION TECHNOLOGY AND ELECTRICAL ENGINEERING\\}
	%{\sffamily\fontsize{9}{1pt}\selectfont TIETO- JA SÄHKÖTEKNIIKAN TIEDEKUNTA\\}
	\vspace{65 mm}
	{\textbf{\fontsize{16}{19pt}\selectfont \getfirstname\ \getlastname }\\}
	\vspace{15 mm}
	{\textbf{\fontsize{18}{22pt}\selectfont OPTIMIZATION OF SPACECRAFT OPERATOR WORKFLOW THROUGH STATE-OF-THE-ART FRONT-END TECHNOLOGIES\\}}
	\vspace{53 mm}
	{\fontsize{14}{17}\selectfont Master's Thesis \\Degree Programme in Computer Science and Engineering \\ June 2017\\}
	%{\fontsize{14}{17}\selectfont Diplomityö \\Tietotekniikan tutkinto-ohjelma \\ Month 20xx\\}
\end{titlepage}

\selectlanguage{english}

\begin{abstract}
Various different software tools are used when planning and executing spacecraft operations. European Space Agency's Cluster II mission utilizes a web portal called "ClusterWeb" for scheduling and overviewing mission events. A new version of this tool has been developed using modern web techologies to improve usability and to make the tool adaptable to other missions in the future. 

A comparative study was done between the old and new tool to find out how usability and reliability were improved. It was found that...

\keywords space operation, software, usability, reliability

\end{abstract}

\selectlanguage{finnish}
\begin{tiivistelma}
Avaruusoperaatioiden suunnittelussa ja toteutuksessa hyödynnetään paljon erilaisia ohjelmistoja. Euroopan Avaruusjärjestön Cluster II- lennossa käytetään "ClusterWeb"- web-portaalia tapahtumien ajoittamiseen ja antamaan yleiskatsaus lennon tapahtumista. Tästä työkalusta on kehitetty uusi versio moderneja web-teknologioita hyödyntäen. Tarkoituksena on ollut parantaa työkalun käytettävyyttä ja mahdollistaa sen hyödyntäminen myös muiden lentojen yhteydessä.

Vanhan ja uuden ClusterWebin välillä tehtiin vertaileva tutkimus jolla pyritttiin selvittämään millä tavoin käytettävyys ja luotettavuus ovat parantuneet. Tuloksista nähdään että...

\avainsanat avaruusoperaatio, ohjelmisto, käytettävyys, luotettavuus
\end{tiivistelma}

\selectlanguage{english}
%\selectlanguage{finnish}

\sisluettelo
%\tableofcontents

\otsake{FOREWORD}
This \LaTeX -template has been used by various people at department
since the late 1990's, and has slowly improved over time.  It is still
somewhat rough at the edges, but hopefully will be helpful in reducing
some of the pain involved in writing a diploma thesis.

Contributors to the template include Mika Korhonen (original author),
Pekka Pietikäinen, Christian Wieser and Teemu Tokola.  If you make any
improvements to this template, please contact ouspg@ee.oulu.fi, and we
will try to include them in further revisions.

The template was updated during the summer of 2013 by Juha Kylmänen.
%\allekirjoitus{Oulu, Finland \today}

\otsake{ABBREVIATIONS}

\setlongtables
\begin{longtable}[l]{p{3cm}p{0.7\textwidth}}

% Add your abbreviations to abbreviations.tex
AOS & aqcuisition of signal\\
CW & Cluster Web \\
FCT & flight control team \\
GS & ground station \\
LOS & loss of signal \\
LTEF & long-term event file \\
LTP & long-term prediction \\
MPS & mission planning system \\
MSPA & multiple spacecraft per aperture \\
OBRQ & observation request \\
OW & OpsWeb \\
PL & payload \\
SSR & solid-state recorder \\
STEF & short-term event file \\
TDA & trackign and data acquisition \\
UEQ & User Experience Questionnaire \\
UI & user interface \\

\end{longtable}
\setcounter{table}{0}

%Johdanto
\chapter{Introduction}
\sivunumerot
Spacecraft operations require a lot of software for planning, operating and analyzing purposes. Mission-critical software that is for example used for controlling spacecraft is often based on decades-old technologies and is subject to very strict requirements, however for non-critical planning and analyzing purposes software can be developed in a much more agile way using modern tools, which allows experimenting with different technologies and ideas quickly with the goal of finding ways to improve the workflow of spacecraft operators with new software.

European Space Agency's Cluster II flight control team utilizes a PHP-based tool called "Cluster Web" for planning ground station passes for each of the four spacecraft in the Cluster II constellation. It is a timeline interface that can visualize different kinds of mission data that is relevant to the mission planners, and allows them to edit and export a schedule that is then used for the actual mission planning system.

While Cluster Web serves its purpose for Cluster II mission planning, there are some usability issues and areas which could be improved. Other missions also have interest in a similar tool. Because Cluster Web is built on aging technologies and not very easily expandable in its architecture, a re-engineering project called "OpsWeb" was started with the aim of creating a generic timeline visualization tool with modern web technologies like Angular 2 and Django.

OpsWeb's architecture is designed from scratch to be as configurable as possible for any potential use case. It is also designed to provide a more fluid user experience.

This thesis aims to answer several questions regarding Cluster Web and OpsWeb. The main research question could be formulated as "How do usability and user experience change between Cluster Web and OpsWeb?". The following chapters will give some context to this research question and then attempt answering it.

First, in chapter \ref{usability_chapter} there is some background research regarding what are usability and user experience, and how they can be measured. Then, in chapter \ref{operations_chapter}, the Cluster II mission is introduced, in particular how the mission planning works and what is the purpose of Cluster Web. Different features and the technical architecture of Cluster Web are overviewed. After that, chapter \ref{cluweb_chapter} is intended to answer why OpsWeb is needed, what features it has, how it works and how it is being developed.

The evaluation of Cluster Web and OpsWeb and its results are described in chapter \ref{evaluation_chapter}. The purpose of the evaluation is to find out if and how certain aspects like efficiency and perceived user experience changed between Cluster Web and OpsWeb, and what areas need to be further improved. 

In chapter \ref{discussion_chapter} the overall results of the OpsWeb project and the evaluation are discussed, and finally in chapter \ref{conclusion_chapter} some final remarks regarding the thesis are made. % ./introduction.tex

\chapter{Spacecraft operations}
% Spacecraft operations

This section gives an overview of the day-to-day spacecraft operations, specifically how Cluster Web fits in the big picture to give some context to its usage.

\section{Cluster II}
Cluster II is a mission studying Earth's magnetosphere and its interaction with solar wind. It consists of four spacecraft which orbit the Earth together in a tetrahedral formation. Their orbits have apogees (highest point) at about 120 000 km above the Earth and perigees (lowest point) at about 20 000 km. The eccentric orbits take the spacecraft through some key areas of scientific interest in the magnetosphere, such as the magnetopause and the bow shock. The formation between the four spacecraft is important for determining how the properties of the magnetosphere change over time and location.

Launched in 2000, the mission was originally planned to run until the end of 2003 but it has been extended several times, as of 2017 the mission will run until the end of 2018. While the spacecraft are aging and some systems are no longer operational, they're still producing useful science data.

\section{Mission planning}
The four Cluster II spacecraft are operated from the European Space Operations Centre (ESOC) in Darmstadt, Germany. The scientific operations are planned by Joint Science Operations Team (JSOC) in Chilton, UK.

JSOC sends an OBRQ (Observation Request) file to the flight control team at ESOC for each weekly planning period (PP). The file includes information about which science instruments should be running on the spacecraft at a given time, and what telemetry and data acquisition (TDA) mode should be used. This is planned based on what region of the magnetosphere the spacecraft is flying through.

Mission planner at ESOC uses Cluster Web to create a pass plan for the planning period. Passes are the times when the spacecraft is in contact with a ground station during which data is dumped from the on-board solid state recorder (SSR) and commands can be sent to the spacecraft. 

Mission planner makes sure that the spacecraft is able to contact a ground station during the passes and has sufficient link budget to downlink data. There are different factors affecting the link budget like the spacecraft's distance and elevation angle, and what TDA mode is being used. For maximum efficiency of data collection it is also important to plan the data dump so that the SSR doesn't fill up or empty completely during a pass, but is dumped at sufficient intervals.

After the pass plan (PlanCWEB) file has been created and exported from Cluster Web, it is imported to mission planning system (MPS) together with OBRQ and other relevant files. MPS generates three different types of files: DSF's, SOR's and EVFM's. DSF'S are command sequences to be linked to the spacecraft  by the spacecraft controllers, SOR's are procedures to be executed by the automation system, and EVFM's are jobs to be executed by the ground stations.

The forementioned files are then transferred to servers from which they will be used for controlling the spacecraft and the ground stations to achieve the planned sequence of passes during the next planning period.

The data collected during the passes is tranferred to various different data centres around the world in the distributed Cluster Science Data System (CSDS).

\section{Cluster Web}
Cluster Web is a web portal that the mission planner uses for planning passes as mentioned in the last section. In this section the features and the technology of the software will be detailed more.

The main interface of Cluster Web is a timeline that offers an overview of the mission plan. The timeline is split vertically into four sections, one for each spacecraft in the mission. Different types of data are visualized on the timeline, some of which events or states with a start and end time, some are events at a specific point in time and some are time-value sets shown as graphs on the timeline.

The following table lists the types of visualized data:

\begin{table}[!ht]
% Add some padding to the table cells:
\def\arraystretch{1.1}%
\begin{center}
  \caption{Types of data visualized in Cluster Web}
  \label{tab:clusterweb_data_table}
  \begin{tabular}{| l | l | l | l | l | }
    \hline
    Name & Visualization & Editable & Type & Source \\
    \hline
    Passes & Rectangles & Yes & Operational & Flight control \\
    Visibilities & Rectangles & No & Environment & Flight dynamics \\
    Bookings & Rectangles & No & Operational & Ground stations \\
    TDA modes & Background & No & Operational & Science operations \\
    SSR fill level & Line graph & Indirectly & Resource & Telemetry/prediction \\
    Payload sequences & Rectangles & No & Operational & Science operations \\
    Mneuver times & Diamonds & No & Operational & Flight dynamics \\
    Apogees/perigees & Small points & No & Environment & Flight dynamics \\
    Eclipses & Background & No & Environment & Flight dynamics \\
    \hline
  \end{tabular}

  \end{center}
\end{table}

The data is aggregated from multiple different sources which are imported into Cluster Web's database periodically to keep it up to date. Ground station passes are the only thing that can be edited by the flight control team, rest of the data can either operational events dictated by other teams, naturally occuring events which depend on the orbit of the spacecraft, or resources on board the spacecraft which are influenced by the operational events.

\subsection{Passes}

\subsection{SSR fill level}

%Pictures in .eps if you use latex, .pdf or .png if you use pdflatex. Don't specify the extension so you can use both!
\begin{figure}[ht]
  \begin{center}
    \includegraphics*[width=1\textwidth]{old_clusterweb}
  \end{center}
  \caption{A screenshot of the Cluster Web timeline interface}
  \label{fig:old_clusterweb}
\end{figure}  % ./operations.tex

\chapter{Re-engineered Cluster Web}
% Re-engineered ClusterWeb Chapter

The re-engineering project of Cluster Web has the aim to...

%Pictures in .eps if you use latex, .pdf or .png if you use pdflatex. Don't specify the extension so you can use both!
\begin{figure}[ht]
  \begin{center}
    \includegraphics*[width=1\textwidth]{old_clusterweb}
  \end{center}
  \caption{A screenshot of the Cluster Web main interface}
  \label{fig:old_clusterweb}
\end{figure}

\section{Requirements}

\section{Development}

\section{Technology}

  % ./cluweb.tex

\chapter{Evaluation}
% Evaluation Chapter

This section covers the usability evaluation done on the old Cluster Web and the new CluWeb. The aim of the evaluation is to justify the need for a re-engineered Cluster Web and to gain qualitative and quantitative data about how the user experience was improved between the different versions.

The evaluation was done in two phases: at first, focusing only on how, why and when operations engineers and spacecraft controllers use the old Cluster Web and what are its strengths and shortcomings. The results can then help in understanding what is required from the new CluWeb to fullfill the users' needs and on what areas in particular should be a focus for improving the user experience.

In the second phase of the evaluation, a comparative evaluation between Cluster Web and CluWeb was performed to find out how it improved in terms of usability. CluWeb wasn't functionally yet on par with the old Culster Web at this point of development as it was missing the pass planning functionality entirely, so the evaluation focused on two aspects which could be compared: monitoring and data visualization.

\section{Methods}

Various different methods exist for evaluating user experience.

\section{Results}  % ./evaluation.tex

\chapter{Discussion}
% Discussion Chapter

Cluster Web and its successor OpsWeb have now been overviewed in this thesis, as well as the evaluation that was done to find out differences between them and to gain feedback for future development. Members of the Cluster Flight Control Team find Cluster Web to be an invaluable tool in their daily jobs; other missions at ESOC do not have the same kind of a interactive schedule visualization tool, and this is why it is important that OpsWeb could be easily applied to other purposes.

Going back to the three research questions defined in section \ref{research_questions}, what needs to be discussed is what differences in usability and user experience were noticed between Cluster Web and OpsWeb in the evaluation, in particular because of new features like improved timeline navigation and timeline configurability.

The evaluation results indicate that OpsWeb allows users to perform tasks faster and with less mouse clicks than Cluster Web, although more evaluation is needed to get better information on how much exactly does OpsWeb improve over its predecessor. An usability issue with OpsWeb's calendar input was discovered which could have caused wrong answers to a certain task, and as such may have lowered the effectiveness of the system. Some problems with the evaluation itself were encountered and noted.

Participants appeared to have a very positive user experience with OpsWeb, and it is a clear improvement over Cluster Web on every scale of the User Experience Questionnaire. In particular the scale "Efficiency" saw a huge improvement, meaning that users found OpsWeb to be a lot faster and practical to use, perhaps due to the mouse-navigable timeline that can be more intuitive to use than having to click and wait every time when zooming or moving the timeframe.

The experience level of the participants was found to not have a significant effect on user experience on OpsWeb. In Cluster Web's case there was only one inexperienced participant, who responded more positively than the average. Perhaps with a larger amount of participants and more tasks it would be possible to shed some light on learnability of Cluster Web and OpsWeb.

The biggest difference between the system architectures of Cluster Web and OpsWeb is the switch from back-end generated HTML to a Javascript-based web application. In Cluster Web, back-end generates the entire visualization every time the timeline is moved, while in OpsWeb, the back-end only provides authentication and access to data; the front-end draws the data visualization and responds to user input in real time. This enables a faster user experience and more flexibility for the developers by keeping changes to front-end and back-end code separate.

Configurability of OpsWeb could be considered to be its most important new feature that could enable it to be used in varying ways in different missions without having to change anything in the source code. It can hopefully allow seamless adoption of the system because the development team will not need to do much custom development for different missions. 

The evaluation of configurations in this thesis did not lead to conclusive results, though some feedback was gathered on potential use cases and improvements. An idea for future research could be to delve deeper into the possibility of visualizing data in different ways and how it changes the way users perceives information. Things like colors, sizes and arrangements of the timelines could be changed and compared.

Any kind of data that specifies time instants, time intervals or a time-value series can be visualized using OpsWeb given that the data is in the correct JSON format. Potential use cases outside spacecraft operations could be for example visualizing TV guides, workplace shift schedules, movie theater showtimes, sleeping habits, etc. It would be interesting for future research to experiment with different use cases and evaluate them. Bringing the system outside of the field of spacecraft operations would make it possible to evaluate with a larger number of participants.

Participants had many interesting ideas on how to improve OpsWeb by making its data visualization clearer and more informative, and by adding the ability to manipulate the data which would allow recreating Cluster Web's pass planning functionality in a more user-friendly manner in OpsWeb. This is one of the things planned for future releases. Another important feature will be making the back-end more generic; users should be able to specify new database tables and insert data conveniently through a user interface without having to edit the back-end code.

Overall, OpsWeb shows promise and will hopefully see use in the future in a wide variety of missions and other use cases. Using modern web technologies and agile development methods can allow teams to build highly usable software relatively quickly to support spacecraft operations.  % ./discussion.tex

\chapter{Summary}
% Summary Chapter

In this section the thesis will be summarized  % ./summary.tex

\bibliographystyle{di}
\bibliography{di}
\end{document}
