%
% Template for Department of Electrical and Information Engineering Diploma Thesis v1.1.2013
% Authors: Mika Korhonen (original author), Pekka Pietikäinen, Christian Wieser, Teemu Tokola and Juha Kylmänen.
% If you make any improvements to this template, please contact ouspg@ee.oulu.fi
%

\documentclass[a4paper, 12pt,titlepage]{dithesis}
\usepackage[english,finnish]{babel}
\usepackage[utf8]{inputenc}
\usepackage[T1]{fontenc}  
\usepackage{times}
\usepackage{tabularx}
\usepackage{graphicx}
\usepackage{float}
\usepackage{enumerate}
\usepackage{placeins}
\usepackage{fancybox}
\usepackage{verbatim}
\usepackage{longtable}
\usepackage{di}
\usepackage[hyphens]{url}
\usepackage{boxedminipage}
\usepackage{subfigure}
\usepackage{multirow}
\usepackage{amsfonts}
\usepackage{xcolor}
\tolerance=500

%\usepackage[a4paper,margin=2.5cm,dvips]{geometry}
%\geometry{papersize={210mm,297mm}}
%dvipdf -sPAPERSIZE=a4

% The following code removes %-signs with URL:s longer than 72 chars
\begingroup
\makeatletter
\g@addto@macro{\UrlSpecials}{%
  \endlinechar=13 \catcode\endlinechar=12
  \do\%{\Url@percent}\do\^^M{\break}}
 \catcode13=12 %
 \gdef\Url@percent{\@ifnextchar^^M{\@gobble}{\mathbin{\mathchar`\%}}}%
\endgroup %

%\selectlanguage{finnish}

\otsikko{Satelliittioperaattoreiden työnkulun optimointi uudenaikaisten web-teknologioiden avulla}
\title{Optimization of spacecraft operator workflow through modern web technologies}

\etunimi{Karri}
\sukunimi{Ojala}
\valvoja{}
\koulutusohjelma{information} % {information | electrical}
\vuosi{2018}
\tyo{Master} % {Bachelor | Master}
\kieli{english} % {finnish | english}

\begin{document}

\begin{titlepage}
	\centering{\includegraphics*[width=0.3\textwidth]{uni_logo}\\}
	{\sffamily\fontsize{9}{1pt}\selectfont FACULTY OF INFORMATION TECHNOLOGY AND ELECTRICAL ENGINEERING\\}
	%{\sffamily\fontsize{9}{1pt}\selectfont TIETO- JA SÄHKÖTEKNIIKAN TIEDEKUNTA\\}
	\vspace{65 mm}
	{\textbf{\fontsize{16}{19pt}\selectfont \getfirstname\ \getlastname }\\}
	\vspace{15 mm}
	{\textbf{\fontsize{18}{22pt}\selectfont OPTIMIZATION OF SPACECRAFT OPERATOR WORKFLOW THROUGH MODERN WEB TECHNOLOGIES\\}}
	\vspace{53 mm}
	{\fontsize{14}{17}\selectfont Master's Thesis \\Degree Programme in Computer Science and Engineering \\ January 2018\\}
	%{\fontsize{14}{17}\selectfont Diplomityö \\Tietotekniikan tutkinto-ohjelma \\ Month 20xx\\}
\end{titlepage}

\selectlanguage{english}

\begin{abstract}
Various different software tools are used when planning and executing spacecraft operations. European Space Agency's Cluster II mission utilizes a web portal called "Cluster Web" for scheduling and overviewing mission events. A new version of this tool called "OpsWeb" is being developed using modern web techologies to improve usability and to make the tool adaptable to other missions in the future. 

A comparative user experience evaluation was done between Cluster Web and OpsWeb to find out how usability was improved. The results indicate that OpsWeb allows users to complete tasks faster and with less effort than Cluster Web, although some issues were discovered. Users also reported that they had a more positive experience using OpsWeb than Cluster Web. Feedback gathered from the evaluation includes some ideas for future development of OpsWeb.

\keywords spacecraft operation, software, usability, user experience, user interface, evaluation, visualization, web

\end{abstract}

\selectlanguage{finnish}
\begin{tiivistelma}
Avaruusoperaatioiden suunnittelussa ja toteutuksessa hyödynnetään paljon erilaisia ohjelmistoja. Euroopan Avaruusjärjestön Cluster II- lennossa käytetään "Cluster Web"- nimistä web-portaalia tapahtumien ajoittamiseen ja antamaan yleiskatsaus lennon tapahtumista. Tästä työkalusta kehitetään uutta "OpsWeb"- nimistä versiota moderneja web-teknologioita hyödyntäen. Tarkoituksena on parantaa työkalun käytettävyyttä ja mahdollistaa sen hyödyntäminen myös muiden lentojen yhteydessä.

Cluster Webin ja OpsWebin välillä tehtiin vertaileva käyttökokemuksen evaluointi jolla pyritttiin selvittämään millä tavoin käytettävyys parantui. Tuloksista nähdään että käyttäjät pystyvät suorittamaan tiettyjä tehtäviä OpsWebillä nopeammin ja helpommin kuin Cluster Webillä, joskin joitain ongelmia havaittiin. Käyttäjät myös ilmaisivat että heidän käyttökokemuksensa oli positiivisempi OpsWebillä kuin Cluster Webillä. Evaluoinnin yhteydessä kerätty palaute antaa joitain ideoita OpsWebin jatkokehitykseen.

\avainsanat avaruusoperaatio, ohjelmisto, käytettävyys, käyttökokemus, käyttöliittymä, evaluointi, visualisaatio, web
\end{tiivistelma}

\selectlanguage{english}
%\selectlanguage{finnish}

\sisluettelo
%\tableofcontents

\otsake{FOREWORD}
I started writing this thesis during an internship at the European Space Operations Centre in Darmstadt, Germany. I worked in OpsWeb development team for six months from May 2017 to October 2017. It was very rewarding to work on something that will hopefully help spacecraft operators in their jobs in the future. Getting a look inside the world of spacecraft operations was also a very unique and interesting experience.

I would like to thank the people at ESOC, especially everyone in the Cluster II Flight Control Team for this opportunity to work with them and write this thesis. I would like to thank my thesis supervisors at University of Oulu, Mourad Oussalah and Matti Pouke, for their feedback and guidance. I would also like to thank my parents for their support.

Karri Ojala, January 2018
%\allekirjoitus{Oulu, Finland \today}

\otsake{ABBREVIATIONS}

\setlongtables
\begin{longtable}[l]{p{3cm}p{0.7\textwidth}}

% Add your abbreviations to abbreviations.tex
AOS & aqcuisition of signal\\
LOS & loss of signal \\
LTP & long-term prediction \\
GS & ground station \\
MSPA & multiple spacecraft per aperture \\
OBRQ & observation request \\
PL & payload \\
LTEF & long-term event file \\
STEF & short-term event file \\
SSR & solid-state recorder \\
TDA & trackign and data acquisition \\
UI & user interface \\


\end{longtable}
\setcounter{table}{0}

%Johdanto
\chapter{Introduction}
\sivunumerot
Spacecraft operations require a lot of software for planning, operating and analyzing purposes. Mission-critical software that is for example used for controlling spacecraft is often based on decades-old technologies and is subject to very strict requirements, however for non-critical planning and analyzing purposes software can be developed in a much more agile way using modern tools, which allows experimenting with different technologies and ideas quickly with the goal of finding ways to improve the workflow of spacecraft operators with new software.

European Space Agency's Cluster II flight control team utilizes a PHP-based tool called "Cluster Web" for plannign ground station passes for each of the four spacecraft in the Cluster II constellation. It is a timeline interface that can visualize different kinds of mission data that is relevant to the mission planners, and allows them to edit and export a schedule that is then used for the actual mission planning system.

While Cluster Web serves its purpose for Cluster II mission planning, there is interest for a similar tool at other missions, and for adding new features which would improve usability. Because Cluster Web is built on aging technologies and not very easily expandable in its architecture, a re-engineering project called "CluWeb" was started with the aim of creating a generic timeline visualization tool with modern web technologies like Angular 2 and Django.

CluWeb's architecture is designed from scratch to be as configurable as possible for any potential use case. It is also designed to provide a more fluid user experience.

This thesis aims to explain what the purpose of Cluster Web is in the context of spacecraft operations, provides an overview to the development process, new features and architecture of CluWeb and does a comparative study between it and Cluster Web to find out how usability was improved. % ./introduction.tex

\chapter{Usability and user experience}\label{usability_chapter}
There are different interpretations on what usability and user experience mean. It is important to understand these words so that a clearer idea can be formed on what needs to be evaluated. This section will give an overview of the state of the art regarding usability and user experience, and how they can be measured.

\section{Meaning of usability and user experience} \label{definitions_section}
The definitions of usability and user experience in the context of software evaluation aren't self-evident, and they can be interpreted in many different ways. However, these terms do have official ISO standards defining them. Bevan et al. \cite{bevanstandard} talk about the importance of using these standards so that the criteria against which software is evaluated stays consistent.

In  ISO FDIS 9241-210 \cite{dis20099241}, usability is defined as "Extent to which  a system, product or service can be used by specified users to achieve specified goals with effectiveness, efficiency and satisfaction in a specified context of use." Therefore, usability seems to have its main focus on the pragmatic goal of achieving some task in a efficient way while using the system, however it also includes the concept of satisfaction.

User experience on the other hand is defined as "A person's perceptions and responses that result from the use and/or anticipated use of a product, system or service." It would appear that user experience is more about deriving pleasure from using a system even if it is not for entertainment purposes. Also worth noting is that this also covers anticipated use, before the user has even seen the system, and how their expectations may compare to reality.

There are many ways to interpret how these terms relate to each other. If the concept of "satisfaction" in the definition of usability is considered to cover "a person's perceptions and responses" in the definition of user experience, then usability can be seen as a umbrella term that covers both the actual measurable work efficiency and personal feelings that stem from using the system.

Väänänen-Vainio-Mattila et al. \cite{vaananen2008towards} and Bevan \cite{bevan2009difference} note that in industry it is often user experience that is used as an umbrella term that includes the work efficiency component of usability, while in research community user experience is seen as the subjective perception that user gets from using a system. Because of its subjective nature, evaluation methods for this academic definition of user experience aren't well-established yet.

In the end there are three different interpretations of user experience according to Bevan \cite{bevan2009difference}: \begin{itemize}
\item An  elaboration  of  the  satisfaction  component  of usability
\item Distinct  from  usability,  which  has  a  historical emphasis on user performance
\item An  umbrella  term  for  all  the  user’s  perceptions  and responses,  whether  measured  subjectively  or objectively
\end{itemize}

Petrie and Bevan \cite{bevanevaluation} further open up the meanings of these terms and their components and how they're used in research community. Usability can contain concepts like learnability, flexibility, memorability, safety and accessibility in addition to efficiency, because all these contribute to achieving some end goal through the use of the system. What exactly is considered usability is dependent on the system in question. 

According to Petrie and Bevan \cite{bevanevaluation}, users may want more from their interaction with the system than just to complete a task efficiently, and this is where user experience is an important aspect to consider. Information technology has become an ubiquitous part of everyday life and is no longer just a means to an end to people.

Tullis and Albert \cite{albert2013measuring} consider user experience to be a broad term that covers the user's entire interaction with the system, including their thoughts, feelings and perceptions. Their view of user experience metrics is that they reveal information about the effectiveness, efficiency and satisfaction that's a result of the interaction between the user and the system.

Hassenzahl et al. \cite{hassenzahl2006user} look at usability and user experience from different points of view and argue that while these terms practically cover the same concepts, they have slightly different focuses. Usability is more concerned with practical task completion and objectively measurable metrics, while user experience tries to balance the practical and hedonic sides of information system usage and also considers how the user feels after using the system.

In Rubin's and Chisnell's \cite{rubin2008handbook} definition, with an usable system "the user can do what he or she wants to do the way he or she expects to be able to do it, without hindrance, hesitation, or questions". An usable system should be "useful, efficient, effective, satisfying, learnable, and accessible". This way of defining usability is similar to others and also covers the concept of user satisfaction. An user is more likely to use a system that generates positive feelings.

In all different definitions, usability seems to cover the basic idea that an user should be able to complete a task using the system efficiently. What is less clear is whether user experience is a part of usability or vice versa, or whether they are two separate concepts. There is support for many different interpretations.

In this thesis, user experience will be used to describe the user's perceived feelings regarding the system. Usability is going to be used to define the more pragmatic task-oriented part of using the system.

% Usability Models

\section{Usability modeling}\label{usability_attributes}
Usability models split the concept of usability into different components which can be linked to some measurable values. Many different concepts like efficiency and learnability were already mentioned in the section \ref{definitions_section}.

Abran et al. \cite{abran2003usability} propose an usability model that enhances upon earlier models and is based on ISO standards. This model includes effectiveness, efficiency, satisfaction, security and learnability. Security is the one term that they added on top of the ISO definition of usability.

There isn't one standard usability model, but many of them build on the ISO definition of usability and cover the same basic ideas. At least the following components can be found across different models and definitions of usability:

\textbf{Usefulness} is a measure of whether the system is actually needed. The task the system is designed for should be something that people need to or want to do, and the system should be helpful in completing that task. One way to measure this would be to ask users to complete a task without a system and then with it, and compare the results. \cite{albert2013measuring, rubin2008handbook}

\textbf{Efficiency} is a measure of how good are the results of using the system relative to the resources it needs. This is often measured in the amount of time required to complete a task, which is possible to evaluate as long as the task is well defined and has a starting and an ending time. Other aspects like the amount of physical and mental effort are also linked to efficiency. \cite{albert2013measuring, rubin2008handbook}

\textbf{Effectiveness} is defined as how well the users can complete tasks using the system. The system should act reliably in a way that the user expects it to in order to be properly usable. Otherwise, the tasks may end up with erroneous results or not completed at all. Comparing the amount of succesfully completed tasks to the amount of failed tasks is one way to measure effectiveness. \cite{albert2013measuring, rubin2008handbook}

\textbf{Learnability} means how quickly and how well the users are able to learn the system so that they can use it effectively. If the system is hard to learn, this will lead to problems in effectiveness and efficiency. Having people use a system for some time and seeing how long it takes for them to use the system effectively can be used as a measure of learnability. \cite{albert2013measuring, rubin2008handbook}

\textbf{Satisfaction} is a hedonic measure that covers the user's feelings about using the system. It is useful to directly ask the users about their opinions and preferences as these can have an effect on perceived usability. At very least, the user shouldn't feel negative emotions like frustration or confusion while and after using the system. Optimally the system would evoke some positive reaction that could encourage the user to interact with the system more. If the system is learnable, efficient and effective, it should be satisfactory to use, but also things like the visual look of the system can have an effect. A questionnaire at the end of evaluation and also asking questions and observing the participant's positive and negative reactions during the evaluation are some ways of collecting data about user satisfaction. \cite{abran2003usability, winter2008comprehensive}

\textbf{Accessibility} often deals with how well the system can be used by people with disabilities, for example poor eyesight or coordination.  \cite{bevanevaluation}

\textbf{Memorability} is  a subset of learnability; how well and how quickly the user can continue using the system after being away from it for some time. \cite{bevanevaluation}

\textbf{Flexibility} means the capability of the system to accommodate to changes desired by the user. \cite{bevanevaluation}

\textbf{Safety} covers the aspects of the system protecting the user from dangerous situations and undesirable conditions. \cite{bevanevaluation, winter2008comprehensive}

\textbf{Security} could be considered to overlap with safety; the system should be able to prevent unauthorized access to its functionality and data. \cite{abran2003usability}

\section{Evaluating usability and user experience}
In this section the available methods of evaluation will be overviewed to get a better understanding on what data should be collected in an evaluation, and what are particularly important aspects to take into account to ensure the validity of the data.

\subsection{Evaluation methods}
According to Rubin and Chisnell \cite{rubin2008handbook}, an usability evaluation has the following basic elements:
\begin{itemize}
\item Develop research questions
\item Use a representative sample of end users
\item Represent the actual work environment
\item Observe the end users using the product
\item Interview the participants
\item Collect quantitative and qualitative performance and preference measures
\item Recommend improvements to design
\end{itemize}

While using real users in an empirical evaluation is time-consuming, is often the best way to identify usability problems. During the development cycle, using usability inspection methods like heuristic evaluation is sometimes used, which involves an expert going through the interface and comparing it against a set of criteria such as Nielsens ten heuristics \cite{Solr-oula.410573, nielsen1995usability}. This is faster and easier than doing full usability evaluation with users, but doesn't identify as many problems.

Hollingsed and Novick \cite{hollingsed2007usability} note that using both empirical evaluation and usability inspection would be the most effective solution that is often overlooked.

User evaluations can be done by remotely collecting data during day-to-day usage, for example by showing a survey to the user at random times, or by inspecting from logs how the users interact with the system. \cite{bevanevaluation} This way it is possible to collect large amounts of data, but it may not be as detailed as in-person evaluation.

Data can be collected concurrently during the user evaluation or after it retrospectively. Van den Haak et al. \cite{van2003retrospective} performed a comparative test between concurrent think-aloud (CTA) and retrospective think-aloud (RTA).

CTA was found to find more task-oriented errors while RTA was good for gaining broader user reactions. There is some debate on whether or not CTA has a negative effect on task performance because talking aloud would distract the user, but some have found it to have a positive effect. When performing a complex task the user may verbalize less. RTA is less susceptible this problem. Task complexity for evaluation should be designed so that the users are required to think, but aren't over-burdened so much that they cannot make verbal observations.

A general comparison of different methods can be seen in table \ref{methods_comparison}.

\begin{table}[!ht]
% Add some padding to the table cells:
\def\arraystretch{1.1}%
    \begin{center}
    \caption{Usability evaluation methods comparison}
    \label{methods_comparison}
    \begin{tabular}{| l | l | l |  l | }
    \hline
    Method & Effort &  Data &  Accuracy  \\
    \hline
    Heuristic evaluation   & Low & General usability problems &  Medium     \\
    Remote evaluation & Medium &  Large amount of information &  Medium     \\
    In-person evaluation  & High & Real user impressions &  Good    \\
    CTA  & High &  Detailed impressions &  Good   \\
    RTA  & High & Broad impressions & Good   \\
    \hline
    \end{tabular}
    \end{center}
\end{table}

\subsection{Participant selection}
Selecting the right participants for the study is a very important aspect of user experience evaluation. The participants should be a representative sample of the target userbase of the system. If the test users are selected incorrectly the test results would not be a useful metric of the user experience in real use. \cite{rubin2008handbook, albert2013measuring}

\subsection{Tasks}
The participants should perform different tasks during an evaluation. These tasks should represent some real use case of the system that is intended to be evaluated, and they should be able to bring out differences in usability between the evaluated system. The tasks should require the users to make use of different functionalities of the system.

If the amount of participants is rather small, every user should perform the same set of tasks, as it may not viable to have a different set of users for each task. This means that the evaluation would use a within-subjects design rather than a between-subjects design. 

One phenomenon that is worth noticing is that when users perform tasks, they gradually learn to use the system. This is why users should perform these tasks in different orders to mitigate the learning effect. This distribution of task order is called counter-balancing. The order in which different systems are shown to the participant can also be alternated. \cite{rubin2008handbook}

\subsection{Metrics}
An evaluation should try to answer research questions related to how usable the system is. To do this, data metrics need to be collected on different components of usability, which were defined in the section \ref{usability_attributes}.

For measuring pragmatic usability components like efficiency and effectiveness, common measures are task completion time and error rate. Sometimes a task may require multiple steps during which the user can make multiple errors. 

It is also possible to measure the input rate e.g. the amount of mouse clicks the user makes while performing a task.

Some tasks can have a pre-defined optimal solution that is as efficient as possible; users may not take this optimal path, so the possible extra steps and mistakes can be observed. \cite{hornbaek2006current}

Measuring hedonic user experience is commonly done by asking the participant a set of questions about their experience. Getting spontaneous and immediate reactions from the user at the end of the evaluation is an important part of gaining understanding about the user experience. The participants shouldn't be forced to analyze small details which they may not properly remember anymore at the end of the evaluation. \cite{laugwitz2008construction}

Hornbaek \cite{hornbaek2006current} notes that there are many already validated questionnaires for measuring usability and user experience, but despite this, studies often create their own questionnaires in ad hoc basis. Some common examples of questionnaires are AttrakDiff \cite{hassenzahl2003attrakdiff}, Questionnaire for User Interaction Satisfaction (QUIS) \cite{chin1988questionnaire}, System Usability Scale (SUS) \cite{brooke1996sus} and User Experience Questionnaire (UEQ) \cite{laugwitz2008construction}.

In a study by Tullis et al. \cite{tullis2004comparison} that compares SUS, QUIS and a few other questionnaires, SUS was found to be the simplest and often most reliable one. UEQ was not in this comparison.

The questions in SUS are as follows, all of them rated on a five-choice scale from strongly disagree to strongly agree:

\begin{enumerate}
\item I think that I would like to use this system frequently
\item I found the system unnecessarily complex
\item I thought the system was easy to use                      
\item I think that I would need the support of a technical person to be able to use this system
\item I found the various functions in this system were well integrated
\item I thought there was too much inconsistency in this system
\item I would imagine that most people would learn to use this system very quickly
\item I found the system very cumbersome to use
\item I felt very confident using the system
\item I needed to learn a lot of things before I could get going with this system 
\end{enumerate}

UEQ is split five different scales altogether containing 26 items which apply particularly for hedonic user experience. The items are opposite adjectives on a scale from 1 to 7, and the order of the polarities is randomized. The scales are as follows:

\textbf{Attractiveness:} the user's overall impression of the system, for example "annoying - enjoyable".

\textbf{Perspiquity:} how easy to understand and clear the participants find the system to use, for example on a range "not understandable - understandable". This overlaps with learnability that was covered in section \ref{usability_attributes}

\textbf{Dependability:} how well the user can trust the system and predict what it does, for example on a scale "unpredictable - predictable". This is linked to how effectively the users can use the system.

\textbf{Efficiency:} this is the perceived efficiency of the system. Also considers how organized the interface is. Scale "impractical - practical" is one item of this.

\textbf{Novelty:} the novelty factor of the system has an effect on how eager people are to use it, and how much they will explore different options. This includes items like "creative - dull" and "conservative - innovative".

\textbf{Stimulation:} how engaged users feel when using the system. "Boring - exciting" is an example of this, as well as "motivating - demotivating".

Perspiquity, dependability and efficiency are task-oriented factors and should show a negative correlation with task completion time and error rate. Novelty and stimulation factors only show a weak correlation according to Laugwitz et al. \cite{laugwitz2008construction}

UEQ is particularly meant for measuring user experience, not only usability. It has been found to be fast and efficient to implement, the drawback being that the information is rather high level and doesn't go into specific details. However, by combining the questionnaire with a concrete usability evaluation it is possible to get more comprehensive results. \cite{schrepp2014applying, rauschenberger2013efficient}

Schrepp et al. \cite{schrepp2017construction} have developed a benchmark for analyzing the results of UEQ. The benchmark was constructed based on 246 product evaluations to find baseline mean values and standard deviations for the different scales. This allowed the creation of different intervals on a scale from "bad" to "excellent".

AttrakDiff is older than UEQ, and also meant for measuring perceived user experience, consisting of several different scales of user experience with seven-step items. The scales are "Attractiveness", "Pragmatic Quality", "Identity" and "Stimulation". These were found to have significant correlations with the scales of UEQ in a validation study. \cite{laugwitz2008construction} The original AttrakDiff is in German, it appears that the English version is only accessible through an online service. \cite{attrakdiff}

UEQ's item pool seemingly overlaps a lot with what is covered in SUS (complexity, consistency, confidence...), while presenting the questions in a simpler format, and also going into more detail than SUS. UEQ is easily available offline and has a convenient benchmark for analyzing the data.

\section{Summary}
The definitions of usability and user experience as well as different methods for evaluating them have now been overviewed in this chapter. Having an understanding of these concepts is important for formulating an evaluation plan. 

It was found out that there isn't a general consensus on the differences between usability and user experience, and there can be overlap between these terms depending on interpretation. For the purposes of this thesis, user experience is considered to be the subjective experience of an user, while usability is based on objective metrics of how a user performs with a system. This interpretation is common in user experience research.

Before a product can be evaluated, the context in which it is used must be understood. Cluster II mission and Cluster Web will be introduced in the next chapter.  % ./usability.tex

\chapter{Spacecraft operations}\label{operations_chapter}
% Spacecraft operations

This section gives an overview of the day-to-day spacecraft operations, specifically how Cluster Web fits in the big picture to give some context to its usage.

\section{Cluster II}
Cluster II is a mission studying Earth's magnetosphere and its interaction with solar wind. It consists of four spacecraft which orbit the Earth together in a tetrahedral formation. Their orbits have apogees (highest point) at about 120 000 km above the Earth and perigees (lowest point) at about 20 000 km. The eccentric orbits take the spacecraft through some key areas of scientific interest in the magnetosphere, such as the magnetopause and the bow shock. The formation between the four spacecraft is important for determining how the properties of the magnetosphere change over time and location.

Launched in 2000, the mission was originally planned to run until the end of 2003 but it has been extended several times, as of 2017 the mission will run until the end of 2018. While the spacecraft are aging and some systems are no longer operational, they're still producing useful science data.

\section{Mission planning}
The four Cluster II spacecraft are operated from the European Space Operations Centre (ESOC) in Darmstadt, Germany. The scientific operations are planned by Joint Science Operations Team (JSOC) in Chilton, UK.

JSOC sends an OBRQ (Observation Request) file to the flight control team at ESOC for each weekly planning period (PP). The file includes information about which science instruments should be running on the spacecraft at a given time, and what telemetry and data acquisition (TDA) mode should be used. This is planned based on what region of the magnetosphere the spacecraft is flying through.

Mission planner at ESOC uses Cluster Web to create a pass plan for the planning period. Passes are the times when the spacecraft is in contact with a ground station during which data is dumped from the on-board solid state recorder (SSR) and commands can be sent to the spacecraft. 

Mission planner makes sure that the spacecraft is able to contact a ground station during the passes and has sufficient link budget to downlink data. There are different factors affecting the link budget like the spacecraft's distance and elevation angle, and what TDA mode is being used. For maximum efficiency of data collection it is also important to plan the data dump so that the SSR doesn't fill up or empty completely, but is dumped at sufficient intervals.

After the pass plan (PlanCWEB) file has been created and exported from Cluster Web, it is imported to mission planning system (MPS) together with OBRQ and other relevant files. MPS generates three different types of files: DSF's, SOR's and EVFM's. DSF'S are command sequences to be linked to the spacecraft  by the spacecraft controllers, SOR's are procedures to be executed by the automation system, and EVFM's are jobs to be executed by the ground stations.

The forementioned files are then transferred to servers from which they will be used for controlling the spacecraft and the ground stations to achieve the planned sequence of passes during the next planning period.

The data collected during the passes is tranferred to various different data centres around the world in the distributed Cluster Science Data System (CSDS).

\section{Cluster Web}
Cluster Web is a web portal that the mission planner uses for planning passes as mentioned in the last section. In this section the features and the technology of the software will be detailed more.


%Pictures in .eps if you use latex, .pdf or .png if you use pdflatex. Don't specify the extension so you can use both!
\begin{figure}[ht]
  \begin{center}
    \includegraphics*[width=1\textwidth]{old_clusterweb}
  \end{center}
  \caption{A screenshot of the Cluster Web main interface}
  \label{fig:old_clusterweb}
\end{figure}  % ./operations.tex

\chapter{Re-engineering Cluster Web}\label{cluweb_chapter}
Cluster Web, detailed in the section \ref{clusterweb_section}, is a very powerful tool for planning and monitoring the operations of Cluster mission. Spacecraft operations engineers and spacecraft controllers use it in daily basis for different tasks.

Even though Cluster Web has proven its usefulness, it has certain usability issues, mainly the slow navigation of the timeline using buttons. It is also based on aging technologies and has the front-end and back-end tightly coupled together. Ever since Cluster Web was developed in 2009, web applications have become much more interactive, and the use of touch screen devices has increased.

Other mission teams at ESA such as ExoMars and Swarm have interest in a tool like Cluster Web for their plannign and monitoring purposes. While making a custom solution based on Cluster Web would be possible in theory, it would require a major rewrite for each mission as the codebase is specific to Cluster and is not built with expandability in mind.

Because of the technical limitations and usability issues, it was decided that a new tool would be built based on modern web technologies and structured in such a way that it would be easy to adapt for other missions in the future. This re-engineered version is called "CluWeb". It is developed by a team consisting of Cluster team members and also members from other teams within ESOC.

The author of this thesis worked on this project for six months during an internship. Upon arrival in May 2017, the development had been going on for a couple of months and the main technologies had been chosen. The author contributed on the development of many different features, mainly focusing on the front-end. 

By the end of the six months in October 2017, CluWeb had most of the functionality needed for schedule monitoring purposes, which is the first release that will be used by the Cluster flight control team.

In this section, CluWeb development project, its improved features and the architecture of the system as of October 2017 will be detailed. The plans for the future development of CluWeb will also be overviewed.

\section{Requirements}
\section{New features}
\subsection{Timeline navigation}
\subsection{Configurations}
\subsection{Live monitoring}
\subsection{Sharing schedules}
\subsection{Uberlog integration}
\section{Architecture}
\subsection{Front-end}
\subsection{Back-end}
\section{Development}
\subsection{Scrum}
\subsection{Continuous integration}
\section{Future development}  % ./cluweb_new.tex

\chapter{Evaluation}\label{evaluation_chapter}
% Evaluation Chapter

This section covers the evaluation done on Cluster Web and CluWeb. The aim of the evaluation is to gain knowledge about how the old and new version compare in terms of user experience and usability and to find possible usability problems. Research questions will be defined, methods will be overviewed and the evaluation plan will be constructed in this section.

Because CluWeb wasn't functionally yet on par with the old Cluster Web at this point of development as it was missing the pass planning functionality entirely, the evaluation focused on the data visualization functionality entirely; how well users can find information using the timeline interface.

There are some differences and new features in CluWeb, and their effect on the user experience should be evaluated. In particular the configurable timelines are a new feature that could have a big effect on how people use Cluster Web, because it allows visualizing data from many different perspectives. This is arguably the largest addition compared to the old Cluster Web. Other features include the mouse-based timeline navigation, Uberlog integration on the timeline, and a moving live mode.

The results will hopefully shed some light on how much CluWeb improves the user experience and opens up new possibilities with its new features. Having information about this could also help with setting goals and priorities for future development, but the study is more summative than formative as doing follow-up evaluations is outside of the scope of this thesis. \cite{albert2013measuring}

\section{Research questions} \label{research_questions}
The following main research question is the one that this thesis aims to answer in a reasonably comprehensive way:

\textbf{"How usable is a modern and configurable timeline visualization of spacecraft operations schedule compared to its predecessor?"}

This research question aims to cover all aspects of the evaluation, however there are a few more specific questions which can be made:

The configurability aspect of CluWeb is the main focus of the evaluation as it is the biggest new feature. In particular what could be interesting is how quickly people find data using different arrangements of timelines. This leads to the research question:

\textit{"How do different timeline visualization configurations affect usability?"}

Another aspect that can reasonably be expected to have a noticeable effect on usability is the ability to pan and zoom the timeline using the mouse. It should make user interaction more direct than clicking on buttons and waiting for the system to respond. This can be formulated into the following research question:

\textit{"How does direct mouse navigation affect the usability of a timeline interface compared to UI button-based navigation?"}

Displaying the Uberlog entries on the timeline could have a positive effect on usability, as there is a visual link between the log entries and time. A research question about this could therefore be formulated as follows:

\textit{"What kind of an effect does the visualization of log entries on a timeline have on usability as opposed to a more traditional list interface?"}

\section{Definitions}
There are different interpretations on what usability and user experience mean. It is important to understand these words so that a clearer idea can be formed on what needs to be evaluated. In this section, an overview is done on what these terms generally refer to.

\subsection{Usability and user experience} \label{definitions_section}
The definitions of usability and user experience in the context of software evaluation aren't self-evident, and they can be interpreted in many different ways. However, these terms do have official ISO standards defining them. Bevan et al.talk about the importance of using these standards so that the criteria against which software is evaluated stays consistent. \cite{bevanstandard}

In  ISO FDIS 9241-210, usability is defined as "Extent to which  a system, product or service can be used by specified users to achieve specified goals with effectiveness, efficiency and satisfaction in a specified context of use." Therefore, usability seems to have its main focus on the pragmatic goal of achieving some task in a efficient way while using the system, however it also includes the concept of satisfaction.

User experience on the other hand is defined as "A person's perceptions and responses that result from the use and/or anticipated use of a product, system or service." It would appear that user experience is more about deriving pleasure from using a system even if it is not for entertainment purposes. Also worth noting is that this also covers anticipated use, before the user has even seen the system, and how their expectations may compare to reality.

There are many ways to interpret how these terms relate to each other. If the concept of "satisfaction" in the definition of usability is considered to cover "a person's perceptions and responses" in the definition of user experience, then usability can be seen as a umbrella term that covers both the actual measurable work efficiency and personal feelings that stem from using the system.

Väänänen-Vainio-Mattila et al. and Bevan note that in industry it is often user experience that is used as an umbrella term that includes the work efficiency component of usability, while in research community user experience is seen as the subjective perception that user gets from using a system. Because of its subjective nature, evaluation methods for this this academic definition of user experience aren't well-established yet. \cite{bevan2009difference, vaananen2008towards}

In the end there are three different interpretations of user experience according to Bevan: \begin{itemize}
\item An  elaboration  of  the  satisfaction  component  of usability
\item Distinct  from  usability,  which  has  a  historical emphasis on user performance
\item An  umbrella  term  for  all  the  user’s  perceptions  and responses,  whether  measured  subjectively  or objectively
\end{itemize}
\cite{bevan2009difference}

Petrie and Bevan further open up the meanings of these terms and their components and how they're used in research community. Usability can be contain concepts like learnability, flexibility, memorability, safety and accessibility in addition to efficiency, because all these contribute to achieving some end goal through the use of the system. What exactly is considered usability is dependent to the system in question. 

According to Petrie and Bevan, users may want more from their interaction with the system than just to complete a task efficiently, and this is where user experience is an important thing to consider. Information technology has become an ubiquitous part of everyday life and is no longer just a means to an end to people.
\cite{bevanevaluation}

\cite{bevan2009difference, bevaniso, bevanevaluation, bevanstandard}

Tullis and Albert consider user experience to be a broad term that covers the user's entire interaction with the system, including their thoughts, feelings and perceptions. Their view of user experience metrics is that they reveal information about the effectiveness, efficiency and satisfaction that's a result of the interaction between the user and the system. \cite{albert2013measuring}

Hassenzahl et al. look at usability and user experience from different points of view and argue that while these terms practically cover the same things, they have slightly different focuses. Usability is more concerned with practical task completion and objectively measurable metrics, while user experience tries to balance the practical and hedonic sides of information system usage and also considers how the user feels after using the system.
\cite{hassenzahl2006user}

In Rubin's and Chisnell's definition, with an usable system "the user can do what he or she wants to do the way he or she expects to be able to do it, without hindrance, hesitation, or questions". An usable system should be "useful, efficient, effective, satisfying, learnable, and accessible". \cite{rubin2008handbook} This way of defining usability is similar to others and also covers the concept of user satisfaction. An user is more likely to use a system that generates positive feelings.

In all different definitions, usability seems to cover the basic idea that an user should be able to complete a task using the system efficiently. What is less clear is whether user experience is a part of usability or vice versa, or whether they are two separate concepts. There is support for many different interpretations.

For consistency's sake and  to not use words interchargeably in this thesis, "user experience" is going to be used as an all-encompassing term that covers every aspect of the user's interaction with the system, from their ability to learn how to use the system and complete tasks with it to what kind of feelings stem from the usage of the system. Usability is going to be used to define the more pragmatic task-oriented part of user experience.

Both pragmatic and hedonic goals of information system usage should be considered, and user experience seems to be a better word for this than just usability.

\subsection{Attributes of user experience}\label{usability_attributes}
The concept of usability and user experience should be split into different subcategories to form an understanding of what kind of questions the evaluation results should be able to answer. If we combine what was covered in section \ref{definitions_section}, at least the following categories can be found:

\textbf{Usefulness} is a measure of whether the system is actually needed. The task the system is designed for should be something that people need to or want to do, and the system should be helpful in completing that task. As Cluster Web has been in active use for many years, its usefulness is well established within its userbase, but this should still be evaluated to better find out what it is used for and how useful people find it for different purposes.It is also important to evaluate the usefulness of CluWeb's new features.

\textbf{Efficiency} is a measure of how good are the results of using the system relative to the resources it needs. This is often measured in the amount of time required to complete a task, which is possible to evaluate as long as the task is well defined and has a starting and an ending time. For Cluster Web, a task could  consist of finding some information about a specific data item at a given time. In terms of user experience, the user can also be asked about their perceived efficiency to see how it compares with actual efficiency.

\textbf{Effectiveness} is defined as how well the users can complete tasks using the system. The system should act reliably in a way that the user expects it to in order to be properly usable. Otherwise, the tasks may end up with erroneous results or not completed at all. Error rate is an usual way of measuring this. For Cluster Web's evaluation, a set of test tasks could be defined with their intended outcomes, and whether or not the users could reach the right outcome would be tested.

\textbf{Learnability} means how quickly and how well the users are able to learn the system so that they can use it effectively. If the system is hard to learn, this will lead to problems in effectiveness. Too long learning period is not efficient either. Because the old Cluster Web's current users have been using it for many years, this cannot be evaluated very well with them, but the new CLuster Web can be. Prior experience of participants should be asked at the beginning of the evaluation.

\textbf{Satisfaction} is a hedonic measure that covers the user's feelings about using the system. It is useful to directly ask the users about their opinions and preferences as these can have an effect on perceived usability. At very least, the user shouldn't feel negative emotions like frustration or confusion while and after using the system. Optimally the system would evoke some positive reaction that could encourage the user to interact with the system more. Things like the visual look of a system can have an effect. A questionnaire at the end of evaluation and also asking questions and observing the participant during the evaluation are some ways of collecting data about user satisfaction.

\textbf{Accessibility} often deals with how well the system can be used by people with disabilities, for example poor eyesight or coordination. To some extent this could be taken into account when testing Cluster Web like the visibility of small icons, but finding test subjects could be relatively hard. This is left outside of the scope of this thesis.

\textbf{Memorability} is  a subset of learnability; how well and how quickly the user can continue using the system after being away from it for some time. Because of the limited timeframe that is available for evaluation, it may not be possible to evaluate this, and it is left outside of the scope of this thesis.

\textbf{Safety} covers the aspects of the system protecting the user from dangerous situations and undesirable conditions. It's not clear if this applies to Cluster Web, as even though it is used extensively, it doesn't play a critical part in spacecraft operations and doesn't interface with any physical actuators. Information security could perhaps be considered a part of safety, but that is outside of the scope of this thesis.

\cite{bevanevaluation, rubin2008handbook, albert2013measuring, laugwitz2008construction}

\section{Formulating an evaluation plan}
In this section the available methods of evaluation will be overviewed to get a better understanding on what data should be collected and how, and what are particularly important things to take into account to ensure the validity of the data.

\subsection{Participant selection}
Selecting the right participants for the study is a very important aspect of user experience evaluation. The participants should be a representative sample of the target userbase of the system. If the test users are selected incorrectly the test results would not be a useful metric of the user experience in real use.

In Cluster Web's case, it is reasonably easy to  define and test the entire userbase as it is the Cluster II flight control team which is around 10 people. These will likely be the  first people to use the new CluWeb. Testing people from outside of the team could be considered as well as it might give more results on learnability.

CluWeb has the potential of attaining a larger userbase, as it could be used for visualizing any type of data which fit the descriptions defined in section \ref{vis_types}. 

A clear example of another userbase would be other ESA missions which have similar needs as Cluster II for visualizing operations schedules. Apart from space missions, it could also be used by ground station teams.

There's also potential use cases entirely unrelated to space industry which could be found. Any kind of data that consists of timed events or values can be visualized using CluWeb.

These other use cases are outside of the scope of this evaluation as it would take some time to make tailored solutions for other clients even though CluWeb's modular and configurable design would make it relatively easy compared to its predecessor.

\cite{rubin2008handbook, albert2013measuring}

\subsection{Evaluation methods}
Software can be evaluated in various different ways. Testing the system with real users in an empirical evaluation is time-consuming but is often the best way to identify usability problems. During the development cycle, using usability inspection methods like heuristic evaluation is sometimes used, which involves an expert going through the interface and comparing it against a set of criteria such as Nielsens ten heuristics \cite{nielsen2005ten, nielsen1995usability}. This is faster and easier than doing full usability evaluation with users, but doesn't identify as many problems.

Hollingsed and Novick note that using both empirical evaluation and usability inspection would be the most effective solution that is often overlooked. \cite{hollingsed2007usability}

During the development of CluWeb, formal usability inspections were not done. It was up to the developers to identify potential usability problems. It is helpful that some members of the development team are future users of the software themselves. An usability inspection such as heuristic evaluation could be performed, but the main focus of this thesis is on empirical user evaluation which should produce more useful and interesting results.

User evaluations can be done by collecting data during day-to-day usage, for example by showing a survey to the user at random times, or by inspecting from logs how the users interact with the system. \cite{bevanevaluation} In this evaluation though, this won't be done as it would require additional development, which could be difficult especially in old Cluster Web's case.

Data can be collected concurrently during the test or after it retrospectively. Van den Haak et al. performed a comparative test between concurrent think-aloud (CTA) and retrospective think-aloud (RTA).

CTA was found to find more task-oriented errors while RTA was good for gaining broader user reactions. There is some debate on whether or not CTA has a negative effect on task performance because talking aloud would distract the user, but some have found it to have a positive effect. When performing a complex task the user may verbalize less. RTA is less susceptible this problem. Task complexity for evaluation should be designed so that the users are required to think, but aren't over-burdened so much that they cannot make verbal observations.
\cite{van2003retrospective}

In this evaluation, participants will be encouraged think aloud about their observations during the evaluation, and those observations will be noted down, and afterwards there will be a retrospective questionnaire.

As the participant count is rather small, the user evaluations can be performed in-person. To minimize the effect of external factors on the results, the evaluation should be performed with all participants in the same setting with the same equipment.

\subsection{Tasks}
The users will perform different tasks during the evaluation. These tasks should represent some real use case of the system that is intended to be evaluated. In this evaluation, the tasks are going to focus on navigating the timeline and finding information by using the timeline data visualization, because these are linked to the research questions defined in section \ref{research_questions}.

Same tasks should be performed with the old and new Cluster Web, and different timeline configurations in the new Cluster Web should be compared as well in how well they work for searching different types of information.

The amount of participants is rather small so every user will perform the same set of tasks, as it is not viable to have a different set of users for each task, and the tasks aren't going to be rather quick to complete anyways. This means that the evaluation will use a within-subjects design rather than a between-subjects design. 

One thing that is worth noticing is that when users perform tasks, they gradually learn to use the system. This is why users should perform these tasks in different orders to mitigate the learning effect. This distribution of task order is called counter-balancing. The order in which the new and old Cluster Web are evaluated can also have a biasing effect on the results, this is why their order will also be distributed equally. Same applies to the order the participants are shown different data visualization configurations in the new Cluster Web.
\cite{rubin2008handbook}

\subsection{Metrics}
The evaluation should try to answer the questions defined in section \ref{research_questions}. To do this, data metrics needs to be collected on different aspects of usability, which were defined in the section \ref{usability_attributes}.

Some preliminary information should be collected from each user that is relevant to the evaluation. In Cluster Web's case, the level of expertise could be one metric. This information could be collected by asking for how long the participant has been using Cluster Web, how often they use it and what is their self-reported level of expertise on a Likert scale. \cite{likert1932technique} Also what could be considered is how much they use Cluster Web's different functionalities, as people may have different needs depending on their job.

Independent variables of the evaluation will be the version of Cluster Web (old/new), different timeline configurations and the participants' self-reported level of experience. The dependent variables of the evaluation could be considered to be something like this:

\begin{itemize}
\item Usefulness of something on a scale 1-10
\item Time to complete a task in seconds
\item Effectiveness of completing a task as a success/failure ratio
\item The amount of effort a task took to complete on a scale 1-10
\item How satisfying something is on a scale 1-10
\end{itemize}

Asking the participant's spontaneous and immediate reactions at the end of the evaluation is important on gaining understanding of the user experience. The participants shouldn't be forced to analyze small details which they may not properly remember anymore at the end of the evaluation. 

The evaluation "item pool" will be formed from opposite adjectives that describe something about the user's experience with the system. These should be symmetrically ranges of numbers from which the participants can choose from. Likert scale is a simple choice for this. \cite{laugwitz2008construction} \cite{likert1932technique}

The questions should always ask just about one thing; there shouldn't be a question that for example asks "How useful and easy to use you found the system?" as these are two separate concepts.

Laugwitz et al. suggest some factors consisting of different items which apply particularly for user experience. These are following:

\textbf{Perspiquity:} how easy to understand and clear the participants find the system to use, for example on a range "difficult to understand - easy to understand". This overlaps with learnability that was covered in section \ref{usability_attributes}

\textbf{Dependability:} how well the user can trust the system and predict what it does. For example on a scale "unpredictable - predictable". This is linked to how effectively the users can use the system.

\textbf{Efficiency:} this is the perceived efficiency of the system. Also considers how organized the interface is. Scale "inefficient - efficient" is one item of this.

\textbf{Novelty:} the novelty factor of the system has an effect on how eager people are to use it, and how much they will explore different options. This includes items like "uninnovative - innovative" and "uncreative - creative".

\textbf{Stimulation:} how engaged users feel when using the system. "Boring - interesting" could be one item.

Perspiquity, dependability and efficiency are task-oriented factors and should show a negative correlation with task completion time and error rate. Novelty and stimulation factors only show a weak correlation according to Laugwitz et al. \cite{laugwitz2008construction}

\section{Evaluation plan}
In this section the finalized evaluation plan will be overviewed, includign the task definitions and the questions which will be asked from the participants.

\subsection{Course of the evaluation}
The evaluation is done one participant at a time, in a closed space like a meeting room. All participants perform the evaluation on the same computer.

At the beginning of the evaluation, the participant is asked a set of background questions about their prior experience level with Cluster Web and CluWeb. This is defined in section \ref{pre-evaluation}

The evaluation is split into two parts; Cluster Web and CluWeb. The order in which they are evaluated is variated to prevent the order from having an effect on the results. This will also have to be done when evaluation CluWeb's different timeline configurations.

During the evaluation, participants will perform tasks which should mimic actual use scenarios as closely as possible. Because CluWeb only has the ability to visualize data and doesn't allow the user to edit the data at this point of the development, the focus will be on the data visualization aspect. 

For CluWeb, various different timeline configurations are evaluated to find out how they can help users to browse different types of data. The participants are used to the old Cluster Web's schedule display that is divided by spacecraft, but other timeline views could also prove to be useful. The configurations which will be evaluated are detailed in section \ref{configurations_section}. 

The time to complete each task will be noted down from start to finish, and after each task the user will be asked how what kind of experience they had while doing that task. The tasks themselves and the collected metrics will be listed in section \ref{tasks_section}.

After the evaluation, the participants are asked about how their overall experience with the evaluation was. 

\subsection{Configurations} \label{configurations_section}

\textbf{Default:} this configuration is the basic view that will be used for direct comparison between Cluster Web and CluWeb. The schedule is split by spacecraft into four timelines, each of them displaying passes, SSR fill level, science modes, visibilities, bookings and apogees/perigees.

\subsection{Tasks} \label{tasks_section}
The tasks for the participants will be defined in this section. Firstly, the tasks which will compare the user experience between Cluster Web and CluWeb's default configuration, both of which have the schedule divided by the spacecrafts into four timelines. Some tasks will be focused on just navigating to a specific date on the timeline, while others will require the participants to read information from the visualized data.

Participants will perform the same type of tasks with both Cluster Web and CluWeb to gain comparable data, but the same exact task should not be given to the same participant twice if the task requires the user to find some information from the timeline, because they could remember the answer. Instead there should be different variations of the same type of task.

The Cluster Web - CluWeb comparison tasks are as follows:
\begin{enumerate}
\item Navigate the timeline to show 2nd of September 2017.
\item Navigate the timeline back to today.
\item How many ground station passes did Cluster 3 have on...
\begin{enumerate}
\item 15th of September 2017?
\item 14th of September 2017?
\end{enumerate}
\item How long is the next...
\begin{enumerate}
\item Cluster 1 pass?
\item Cluster 4 pass?
\end{enumerate}
\item When does the next...
\begin{enumerate}
\item Cluster 2 pass begin?
\item Cluster 3 pass begin?
\end{enumerate}
\end{enumerate}

\subsection{Pre-evaluation questionnaire} \label{pre-evaluation}

\subsection{Post-evaluation questionnaire} \label{post-evaluation}

\section{Results}
In this section the results of the evaluation will be presented.


\cite{bevanevaluation, rubin2008handbook, albert2013measuring}  % ./evaluation.tex

\chapter{Discussion}\label{discussion_chapter}
% Discussion Chapter

Cluster Web and its successor OpsWeb have now been overviewed in this thesis, as well as the evaluation that was done to find out differences between them and to gain feedback for future development. Members of the Cluster Flight Control Team find Cluster Web to be an invaluable tool in their daily jobs; other missions do not have same kind of interactive schedule visualization tools, and this is why it is important that OpsWeb could be easily applied to other purposes.

The evaluation results indicate that OpsWeb allows users to perform tasks faster and with less mouse clicks than Cluster Web, although more evaluation is needed to get better information on how much exactly does OpsWeb improve over its predecessor. An usability issue with OpsWeb's calendar input was discovered which could have caused wrong answers to a certain task. Some problems with the evaluation itself were encountered and noted.

Participants appeared to have a very positive user experience with OpsWeb, and it is a clear improvement over Cluster Web on every scale of the User Experience Questionnaire. In particular the scale "Efficiency" saw a huge improvement, meaning that users found OpsWeb to be a lot faster and practical to use, perhaps due to the mouse-navigable timeline that can be more intuitive to use than having to click and wait every time when zooming or moving the timeframe.

The experience level of the participants was found to not have a significant effect on user experience on OpsWeb. In Cluster Web's case there was only one inexperienced participant, who responded more positively than the average.

The biggest difference between the system architectures of Cluster Web and OpsWeb is the switch from back-end generated HTML to a Javascript-based web application. In Cluster Web, back-end generates the entire visualization every time the timeline is moved, while in OpsWeb, the back-end only provides authentication and access to data; the front-end draws the data visualization and responds to user input in real time. This enables a faster user experience and more flexibility for the developers by keeping changes to front-end and back-end code separate.

Configurability of OpsWeb could be considered to be its most important new feature that could enable it to be used in varying ways in different missions without having to change anything in the source code. It can hopefully allow seamless adoption of the system because the development team will not need to do much custom development for different missions. 

Any kind of data that specifies time instants, time intervals or a time-value series can be visualized using OpsWeb given that the data is in the correct JSON format. Potential use cases outside spacecraft operations could be for example visualizing TV guides, workplace shift schedules, movie theater showtimes, sleeping habits, etc.

The evaluation of configurations in this thesis did not lead to conclusive results, though some feedback was gathered on potential use cases and improvements. An idea for future research could be to delve deeper into the possibility of visualizing data in different ways and how it changes the way users perceives information. Things like colors, sizes and arrangements of the timelines could be changed and compared.

Participants had many interesting ideas on how to improve OpsWeb by making its data visualization clearer and more informative, and by adding the ability to manipulate the data which would allow recreating Cluster Web's pass planning functionality in a more user-friendly manner in OpsWeb. This is one of the things planned for future releases. Another important feature will be making the back-end more generic; users should be able to specify new database tables and insert data conveniently through a user interface without having to edit the back-end code.

Overall, OpsWeb shows promise and will hopefully see use in the future in a wide variety of missions and other use cases. Using modern web technologies and agile development methods can allow teams to build highly usable software relatively quickly to support spacecraft operations.  % ./discussion.tex

\bibliographystyle{di}
\bibliography{di}
\end{document}
