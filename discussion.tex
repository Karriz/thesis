% Discussion Chapter

Cluster Web and its successor OpsWeb have now been overviewed in this thesis, as well as the evaluation that was done to find out differences between them and to gain feedback for future development. Members of the Cluster Flight Control Team find Cluster Web to be an invaluable tool in their daily jobs; other missions do not have same kind of interactive schedule visualization tools, and this is why it is important that OpsWeb could be easily applied to other purposes.

The evaluation results indicate that OpsWeb allows users to perform tasks faster and with less mouse clicks than Cluster Web, although more evaluation is needed to get better information on how much exactly does OpsWeb improve over its predecessor. An usability issue with OpsWeb's calendar input was discovered which could have caused wrong answers to a certain task, and as such may have lowered the effectiveness of the system. Some problems with the evaluation itself were encountered and noted.

Participants appeared to have a very positive user experience with OpsWeb, and it is a clear improvement over Cluster Web on every scale of the User Experience Questionnaire. In particular the scale "Efficiency" saw a huge improvement, meaning that users found OpsWeb to be a lot faster and practical to use, perhaps due to the mouse-navigable timeline that can be more intuitive to use than having to click and wait every time when zooming or moving the timeframe.

The experience level of the participants was found to not have a significant effect on user experience on OpsWeb. In Cluster Web's case there was only one inexperienced participant, who responded more positively than the average. Perhaps with a larger amount of participants and more tasks it would be possible to shed some light on learnability of Cluster Web and OpsWeb.

The biggest difference between the system architectures of Cluster Web and OpsWeb is the switch from back-end generated HTML to a Javascript-based web application. In Cluster Web, back-end generates the entire visualization every time the timeline is moved, while in OpsWeb, the back-end only provides authentication and access to data; the front-end draws the data visualization and responds to user input in real time. This enables a faster user experience and more flexibility for the developers by keeping changes to front-end and back-end code separate.

Configurability of OpsWeb could be considered to be its most important new feature that could enable it to be used in varying ways in different missions without having to change anything in the source code. It can hopefully allow seamless adoption of the system because the development team will not need to do much custom development for different missions. 

The evaluation of configurations in this thesis did not lead to conclusive results, though some feedback was gathered on potential use cases and improvements. An idea for future research could be to delve deeper into the possibility of visualizing data in different ways and how it changes the way users perceives information. Things like colors, sizes and arrangements of the timelines could be changed and compared.

Any kind of data that specifies time instants, time intervals or a time-value series can be visualized using OpsWeb given that the data is in the correct JSON format. Potential use cases outside spacecraft operations could be for example visualizing TV guides, workplace shift schedules, movie theater showtimes, sleeping habits, etc. It would be interesting for future research to experiment with different use cases and evaluate them. Bringing the system outside of the field of spacecraft operations would make it possible to evaluate with a larger number of participants.

Participants had many interesting ideas on how to improve OpsWeb by making its data visualization clearer and more informative, and by adding the ability to manipulate the data which would allow recreating Cluster Web's pass planning functionality in a more user-friendly manner in OpsWeb. This is one of the things planned for future releases. Another important feature will be making the back-end more generic; users should be able to specify new database tables and insert data conveniently through a user interface without having to edit the back-end code.

Overall, OpsWeb shows promise and will hopefully see use in the future in a wide variety of missions and other use cases. Using modern web technologies and agile development methods can allow teams to build highly usable software relatively quickly to support spacecraft operations.