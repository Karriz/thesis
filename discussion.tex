% Discussion Chapter

Cluster Web and its successor OpsWeb have now been overviewed in this thesis, as well as the evaluation that was done to find out differences between them and to gain feedback for future development. Members of the Cluster Flight Control Team find Cluster Web to be an invaluable tool in their daily jobs; other missions do not have same kind of interactive schedule visualization tools, and this is why it is important that OpsWeb could be easily applied to other purposes.

It was found out that OpsWeb allows users to perform tasks faster and with less effort than Cluster Web, although more evaluation is needed to get information on how much exactly does OpsWeb improve over its predecessor. Some problems with the evaluation were encountered and noted.

Participants had a very positive user experience with OpsWeb, and it is a clear improvement over Cluster Web on every scale of the User Experience Questionnaire. In particular the scale "Efficiency" saw a huge improvement, meaning that users found OpsWeb to be a lot faster and practical to use, perhaps due to the mouse-navigable timeline that can be more intuitive to use than having to click and wait every time when zooming or moving the timeframe.

The biggest difference in the system architecture of Cluster Web and OpsWeb is the switch from back-end generated HTML to a Javascript-based web application. The back-end provides authentication and access to data; the front-end draws the data visualization and responds to user input in real time. This enables a faster user experience and more flexibility for developers.

Configurability of OpsWeb could be its most important new feature that could enable it to be used in varying ways in different missions without having to change anything in the source code. It can hopefully allow seamless adoption of the system because the development team will not need to do much custom development for different missions.

The evaluation of configurations in this thesis did not lead to conclusive results. An idea for future research could be to delve deeper into the possibility of visualizing data in different ways and how it changes the way users perceives information. Things like colors, sizes and arrangements of the timelines could be changed and compared.

Participants had many interesting ideas on how to improve OpsWeb by making its data visualization clearer and more informative, and by adding the ability to manipulate the data.

Overall, OpsWeb shows promise and will hopefully see use in the future in a wide variety of missions and other applications. Using modern web technologies and agile development methods can allow teams to build highly usable software relatively quickly to support different activities.