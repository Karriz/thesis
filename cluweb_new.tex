Cluster Web, detailed in the section \ref{clusterweb_section}, is a very powerful tool for planning and monitoring the operations of Cluster mission. Spacecraft operations engineers and spacecraft controllers use it in daily basis for different tasks.

Even though Cluster Web has proven its usefulness, it has certain usability issues, mainly the slow navigation of the timeline using buttons. It is also based on aging technologies and has the front-end and back-end tightly coupled together. Ever since Cluster Web was developed in 2009, web applications have become much more interactive, and the use of touch screen devices has increased.

Other mission teams at ESA such as ExoMars and Swarm have interest in a tool like Cluster Web for their plannign and monitoring purposes. While making a custom solution based on Cluster Web would be possible in theory, it would require a major rewrite for each mission as the codebase is specific to Cluster and is not built with expandability in mind.

Because of the technical limitations and usability issues, it was decided that a new tool would be built based on modern web technologies and structured in such a way that it would be easy to adapt for other missions in the future. This re-engineered version is called "OpsWeb". It is developed by a team consisting of Cluster team members and also members from other teams within ESOC.

The author of this thesis worked on this project for six months during an internship. Upon arrival in May 2017, the development had been going on for a couple of months and the main technologies had been chosen. The author contributed on the development of many different features, mainly focusing on the front-end. 

By the end of the six months in October 2017, OpsWeb had most of the functionality needed for schedule monitoring purposes, which is the first release that will be used by the Cluster flight control team.

In this section, OpsWeb development project, its improved features and the architecture of the system as of October 2017 will be detailed. The plans for the future development of OpsWeb will also be overviewed.

\section{Requirements}
As mentioned earlier, the main driving factors for the development of OpsWeb are improving usability and the need for an adaptable tool that could be applied to other missions.
\subsection{Usability} \label{usability_req}
The main user-reported usability issues with Cluster Web are related to the slow responsiveness of the timeline and complicated workflows. 

To navigate the timeline in Cluster Web, the user needs to press buttons in the navigation bar to move the timeline left or right or to change the zoom level in intervals. Once the button has been pressed, it can take a few seconds for the data visualization to reload in the new time range. There is a calendar menu that allows the user to jump straight to a specific date, but when users want to just move the timeline a few days in either direction or zoom out, they use the navigation buttons which are slow. There should be a faster and more intuitive way of navigating the timeline, preferably compatible with modern touch-screen devices.

Another thing that was mentioned is the complexity of adding or editing passes. It is necessary to go through different menus and fill inputs before the pass visually appears on the timeline. An alternative solution suggested by users is having the ability to drop passes directly on the timeline and drag them around to position them.

\subsection{Configurability} \label{configurability_req}
Other missions interested in Cluster Web have similar requirements with displaying different operations events and telemetry data on a timeline. The data types can be split into three categories based on their characteristics: time intervals, time instants and time series.

A time interval is an event with a start and an end. A ground station pass is one example of this. Intervals can be visualized as rectangles which have their left edge at the start time and the right edge at the end time.

Time instants are events which happen at a specific time but have no duration, for example an apogee crossing. They can be visualized with any kind shape or icon at a specific point on the timeline.

Time series data, such as the SSR fill level, is a series of time, value pairs. It can be displayed as a line graph where x-axis is time and y-axis is the range of possible values.

Being able to display any data that fits the aforementioned three categories is one of the requirements for OpsWeb. Users should be able to choose what data they want to display and where the data comes from, how the visualizations are arranged on different timelines and how they look like. Achieving this requires a highly configurable front-end where the individual data visualizations aren't hard-coded, but they are defined by the user in some kind of a configuration file that is then used to build the visualization.

On the back-end side, OpsWeb should also allow users to define their data models and to edit database entries without editing the source code. If some kind of mission-specific data processing is required, it should be plugged in the back-end architecture in a modular way.

\subsection{Openness}
The eventual goal is to make OpsWeb open-source, at first within ESA and later to outsiders. As a generic timeline visualization tool, it could be applied to any kind time-based data visualizations, such as workplace shift schedules or TV guides. In order to make OpsWeb open-source, it needs to reach a release that is no longer subject to big changes in architecture. As of November 2017, OpsWeb is still being developed within the Cluster II Flight Control Team, with a few members from other teams at ESOC also being involved. 

One example of a successful open-source project for space operations is NASA's OpenMCT (Open Mission Control Technologies). Like OpsWeb, it is a web-based data visualization tool but while OpsWeb is focused on timelines, OpenMCT is a more generic dashboard for overviewing and analyzing any kind of technical or scientific real-time data. Users link data to different widgets and place them on layouts. 

OpenMCT is planned to be used for future Lunar rover operations and it is already in use for Jason-3 Earth observation mission. As an open-source project, OpenMCT is not limited to only spacecraft operations, but can be used for any kind of activity that requires displaying data that is being produced in real-time, such as monitoring Internet Of Things- devices. Anyone can download the source code and add features to it. \cite{trimble2014reconfigurable, trimble2016open}

\section{New features}
OpsWeb has a number of new features intended to improve its usability over CluWeb and to allow new possibilities. The new features present in the first release of OpsWeb will be detailed in this section.

\subsection{Timeline navigation}
As mentioned earlier in the section \ref{usability_req}, the navigation of the timeline in Cluster Web is slow. To fix this problem, users can pan and zoom the timeline with mouse in OpsWeb. This method is used for example in map applications such as Google Maps where a 2D plane can be dragged around with the mouse by holding the left button and zoomed in and out with the mouse wheel. In OpsWeb, navigation only happens in x- axis, but otherwise the same idea applies. It also works with touch screen devices.

The schedule timeframe selection calendar is slightly different in OpsWeb; instead of just selecting one day and centering the visible timeframe on that at the current zoom level, the user can choose the first and last days of the preferred timeframe.

\subsection{Configurations}

\subsection{Live monitoring}
Cluster Web is used for monitoring operations in the control room. When opening the page, it loads a view displaying the schedule for the current day and the days before and after it, however, as time goes by, the page needs to be refreshed to get the current time back to the center of the screen.

In OpsWeb, there is a new feature to help with this, called live mode. When the live mode is enabled by clicking a button in the navigation bar, the schedule moves to the current time so that a configured amount of time is displayed in the past and the future. The schedule timeframe is updated periodically so that the line marking the current time stays fixed on the screen while the schedule moves behind it.

Another feature helping with live monitoring is the countdown clock. It displays how much time there is until the next data item of a data type, for example the next pass. A line is drawn from now line to the next item, and a countdown timer is displayed with it. The format of the countdown can be configured with string placeholders.

Live mode and the countdown visualization of OpsWeb were implemented by the author.

\subsection{Sharing schedules}
The schedule timeframe is deep-linked in the URL visible in the browser's address bar. For example, the URL can look like this:

http://cluweb.go.esa.int/schedule?from=2017-10-20T22:00:00\&to=2017-10-21T12:30:00

The "from" and "to" parameters define the beginning and the end time of the visible timeframe of the schedule. They are ISO 8601- formatted date strings.


\subsection{Uberlog integration}
\section{Architecture}
\subsection{Front-end}
\subsection{Back-end}
\section{Development}
\subsection{Scrum}
\subsection{Continuous integration}
\section{Future development}