Cluster Web, detailed in the section \ref{clusterweb_section}, is a very powerful tool for planning and monitoring the operations of Cluster mission. Spacecraft operations engineers and spacecraft controllers use it in daily basis for different tasks.

Even though Cluster Web has proven its usefulness, it has certain usability issues, mainly the slow navigation of the timeline using buttons. It is also based on aging technologies and has the front-end and back-end tightly coupled together. Ever since Cluster Web was developed in 2009, web applications have become much more interactive, and the use of touch screen devices has increased.

Other mission teams at ESA such as ExoMars and Swarm have interest in a tool like Cluster Web for their plannign and monitoring purposes. While making a custom solution based on Cluster Web would be possible in theory, it would require a major rewrite for each mission as the codebase is specific to Cluster and is not built with expandability in mind.

Because of the technical limitations and usability issues, it was decided that a new tool would be built based on modern web technologies and structured in such a way that it would be easy to adapt for other missions in the future. This re-engineered version is called "CluWeb". It is developed by a team consisting of Cluster team members and also members from other teams within ESOC.

The author of this thesis worked on this project for six months during an internship. Upon arrival in May 2017, the development had been going on for a couple of months and the main technologies had been chosen. The author contributed on the development of many different features, mainly focusing on the front-end. 

By the end of the six months in October 2017, CluWeb had most of the functionality needed for schedule monitoring purposes, which is the first release that will be used by the Cluster flight control team.

In this section, CluWeb development project, its improved features and the architecture of the system as of October 2017 will be detailed. The plans for the future development of CluWeb will also be overviewed.

\section{Requirements}
\section{New features}
\subsection{Timeline navigation}
\subsection{Configurations}
\subsection{Live monitoring}
\subsection{Sharing schedules}
\subsection{Uberlog integration}
\section{Architecture}
\subsection{Front-end}
\subsection{Back-end}
\section{Development}
\subsection{Scrum}
\subsection{Continuous integration}
\section{Future development}