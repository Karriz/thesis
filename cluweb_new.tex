Cluster Web, detailed in the section \ref{clusterweb_section}, is a very powerful tool for planning and monitoring the operations of Cluster mission. Spacecraft operations engineers and spacecraft controllers use it in daily basis for different tasks.

Even though Cluster Web has proven its usefulness, it has certain usability issues, mainly the slow navigation of the timeline using buttons. It is also based on aging technologies and has the front-end and back-end tightly coupled together because data visualizations are generated on the back-end as seen in the figure \ref{fig:architecture_cw}. Ever since Cluster Web was developed in 2009, web applications have become much more interactive, and the use of touch screen devices has increased. In a modern web application stack, application logic is executed directly in the browser with Javascript and the back-end provides access to data through a generic REST interface.

Other mission teams at ESA such as ExoMars and Swarm have interest in a tool like Cluster Web for their planning and monitoring purposes. While making a custom solution based on Cluster Web would be possible in theory, it would require a major rewrite for each mission as the codebase is specific to Cluster and is not built with expandability in mind.

\begin{table}[!ht]
% Add some padding to the table cells:
\def\arraystretch{1.1}%
    \begin{center}
    \caption{Usability issues and technical limitations of Cluster Web}
    \label{clusterweb_problems}
    \begin{tabular}{| l | l | }
    \hline
    Problem & Type  \\
    \hline
    No separation between front-end and back-end    &  Technical       \\
    Cluster- specific features hard-coded    &  Technical       \\
    Code is fragmented in many places    &  Technical       \\
    Zooming and moving the timeline is slow    &  Usability       \\
    Menu structure is confusing    &  Usability       \\
    Users cannot configure the visualization    &  Usability       \\
    \hline
    \end{tabular}
    \end{center}
\end{table}

Because of the technical limitations and usability issues listed in table \ref{clusterweb_problems}, it was decided that a new tool would be built based on modern web technologies and structured in such a way that it would be easy to adapt for other missions in the future. This re-engineered version is called "OpsWeb". It is developed by a team consisting of Cluster team members and also members from other teams within ESOC. The author of this thesis contributed in the development of OpsWeb primarily on the front-end during a six-month internship at ESOC.

In this section, OpsWeb development project, its improved features and the architecture of the system as of October 2017 will be detailed. The plans for the future development of OpsWeb will also be overviewed.

 %Pictures in .eps if you use latex, .pdf or .png if you use pdflatex. Don't specify the extension so you can use both!
\begin{figure}[ht]
  \begin{center}
    \includegraphics*[width=1\textwidth]{new_cw}
  \end{center}
  \caption{Cluster Web's default view recreated in OpsWeb}
  \label{fig:new_cw}
\end{figure}

\section{Requirements}
As mentioned earlier, the main driving factors for the development of OpsWeb are improving usability and the need for an adaptable tool that could be applied to other missions.
\subsection{Usability} \label{usability_req}
The main user-reported usability issues with Cluster Web are related to the slow responsiveness of the timeline and complicated workflows. 

To navigate the timeline in Cluster Web, the user needs to press buttons in the navigation bar to move the timeline left or right or to change the zoom level in intervals. Once the button has been pressed, it can take a few seconds for the data visualization to reload in the new time range. This is because the entire visualization is being regenerated in the PHP back-end and the browser just shows the resulting HTML.

There is a calendar menu that allows the user to jump straight to a specific date, but when users want to just move the timeline a few days in either direction or zoom out, they use the navigation buttons which are slow. There should be a faster and more intuitive way of navigating the timeline, preferably compatible with modern touch-screen devices.

Another issue that was mentioned is the complexity of adding or editing passes. It is necessary to go through different menus and fill inputs before the pass visually appears on the timeline. An alternative solution suggested by users is having the ability to drop passes directly on the timeline and drag them around to position them.

\subsection{Configurability} \label{configurability_req}
Other missions interested in Cluster Web have similar requirements with displaying different operations events and telemetry data on a timeline. The data types can be split into three categories based on their characteristics: time intervals, time instants and time series.

A time interval is an event with a start and an end. A ground station pass is one example of this. Intervals can be visualized as rectangles which have their left edge at the start time and the right edge at the end time.

Time instants are events which happen at a specific time but have no duration, for example an apogee crossing. They can be visualized with any kind of shape or icon at a specific point on the timeline.

Time series data, such as the SSR fill level, is a series of time/value pairs. It can be displayed as a line graph where x-axis is time and y-axis is the range of possible values.

Being able to display any data that fits the aforementioned three categories is one of the requirements for OpsWeb. Users should be able to choose what data they want to display and where the data comes from, how the visualizations are arranged on different timelines and how they look like. Achieving this requires a highly configurable front-end where the individual data visualizations aren't hard-coded, but they are defined by the user in some kind of a configuration file that is then used to build the visualization. 

OpsWeb should make use of a separated client-server architecture. The back-end should provide a REST interface for requesting JSON-formatted data, from which the front-end running in the browser could generate a visualization. This would allow much more flexibility for changing the visualization parameters and different components of the system.

On the back-end side, OpsWeb should also allow users to define their data models and to edit database entries without editing the source code. If some kind of mission-specific data processing is required, it should be plugged in the back-end architecture in a modular way.

\subsection{Openness}
The eventual goal is to make OpsWeb open-source, at first within ESA and later to outsiders. As a generic timeline visualization tool, it could be applied to any kind time-based data visualizations, such as workplace shift schedules or TV guides. In order to make OpsWeb open-source, it needs to reach a release that is no longer subject to big changes in architecture. As of November 2017, OpsWeb is still being developed within the Cluster II Flight Control Team, with a few members from other teams at ESOC also being involved. 

One example of a successful open-source project for space operations is NASA's OpenMCT (Open Mission Control Technologies). Like OpsWeb, it is a web-based data visualization tool but while OpsWeb is focused on timelines, OpenMCT is a more generic dashboard for overviewing and analyzing any kind of technical or scientific real-time data. Users link data to different widgets and place them on layouts. While OpenMCT also has a timeline widget, it is much more limited than Cluster Web and would not allow overlaying different types of elements and graphs without modifications. An example screenshot of OpenMCT can be seen in figure \ref{fig:openmct}.

OpenMCT is planned to be used for future Lunar rover operations and it is already in use for Jason-3 Earth observation mission. As an open-source project, OpenMCT is not limited to only spacecraft operations, but can be used for any kind of activity that requires displaying data that is being produced in real-time, such as monitoring Internet Of Things- devices. Anyone can download the source code and add features to it. \cite{trimble2014reconfigurable, trimble2016open}

 %Pictures in .eps if you use latex, .pdf or .png if you use pdflatex. Don't specify the extension so you can use both!
\begin{figure}[ht]
  \begin{center}
    \includegraphics*[width=1\textwidth]{openmct}
  \end{center}
  \caption{A screenshot of NASA's OpenMCT example dashboard}
  \label{fig:openmct}
\end{figure}

\section{New features}
OpsWeb has a number of new features intended to improve its usability over Cluster Web and to allow new possibilities. The new features present in the first release of OpsWeb will be detailed in this section.

\subsection{Timeline navigation} \label{opsweb_nav}
As mentioned earlier in the section \ref{usability_req}, the navigation of the timeline in Cluster Web is slow. To fix this problem, users can pan and zoom the timeline with mouse in OpsWeb. This method is used for example in map applications such as Google Maps where a 2D plane can be dragged around with the mouse by holding the left button and zoomed in and out with the mouse wheel. In OpsWeb, navigation only happens in x- axis, but otherwise the same idea applies. It also works with touch screen devices.

The schedule timeframe selection calendar is slightly different in OpsWeb; instead of just selecting one day and centering the visible timeframe on that at the current zoom level, the user can choose the first and last days of the preferred timeframe.

\subsection{Configurations} \label{opsweb_config}
One of the main driving factors for the development of OpsWeb is configurability, as mentioned in the section \ref{configurability_req}. The definitions of the different elements which are displayed on the schedule should not be hard-coded; users should be able to choose in a flexible manner where they get data from and how it should be displayed on the schedule.

OpsWeb constructs its schedule visualizations using a JSON file that defines the data sources, visualizations and the timeline hierarchy. This file can be edited by clicking a button in the top right corner of the page, which opens a dialog in which the user can edit, save and delete configurations and apply a configuration, which then closes the window and redraws the schedule view to represent that configuration.

The three different categories of data mentioned in the section \ref{configurability_req} are supported in OpsWeb. All data should come in the form of a JSON object array. A data object with the type "interval" is expected to have parameters "time\_start" and "time\_end". An "instant" data object is expected to have a parameter "time". "Series" data should consist of an array of objects with both "time" and some numerical value.

An example timeline configuration with just a single timeline showing Cluster 1 passes would look as follows:

\begin{verbatim}
{
  "name": "CLU1 Pass Schedule",
  "height": "auto",
  "width": "auto",
  "showLabels": true,
  "liveMode": {
    "pastMargin": "1d",
    "futureMargin": "2d"
  },
  "zoom": {
    "min": "2h",
    "max": "12d"
  },
  "data": [
    {
      "type": "interval",
      "name": "pass",
      "provider": "cluweb",
      "source": "passes",
      "domainIdentifier": "$spacecraft",
      "zIndex": 4,
      "showStart": true,
      "showEnd": true,
      "label": "$groundstation",
      "height": 0.5
    }
  ],
  "timelines": [
    {
      "domain": [
        "CLU1"
      ],
      "label": "Cluster 1 Passes",
      "weight": 100,
      "dataRef": [
        {
          "name": "pass"
        }
      ]
    }
  ]
}
\end{verbatim}

What can be seen in this example configuration is that there are two main sections defining the timeline visualization: "data" array that contains the definitions of different data types, and the "timelines" array that contains the timeline hierarchy.

In this case, there is only a single data type, "pass", from a "passes" endpoint of the "cluweb" API, and it is of the type "interval", meaning that each data item should have parameters time\_start and time\_end. Label for each visualized item should come from the parameter "groundstation", and the elements should be placed on the timeline that matches the parameter "spacecraft".

In the timelines array, it can be seen that there is a single timeline with the domain "CLU1", and it references to the data type "pass". This means that all passes of spacecraft "CLU1" will be placed in this timeline.

This structuring allows a lot of flexibility for the users to change their timeline arrangements. For example, it is not necessary to place passes on the timeline by the "spacecraft" domain; instead, "groundstation" could be used as a domain so that each ground station could have its own timeline, displaying passes from all spacecraft on that ground station, as can be seen in the figure \ref{fig:new_cw_estrack}.

 %Pictures in .eps if you use latex, .pdf or .png if you use pdflatex. Don't specify the extension so you can use both!
\begin{figure}[ht]
  \begin{center}
    \includegraphics*[width=1\textwidth]{new_cw_estrack}
  \end{center}
  \caption{"Estrack" timeline configuration, in which each ground station has its own timeline}
  \label{fig:new_cw_estrack}
\end{figure}

The visual style of OpsWeb can also be configured with style configurations. It is a CSS file that can be edited in another tab in the configuration dialog. This allows changing things like colors of the UI elements, visualization elements and different fonts. For example a dark theme can be created with colors easier for the eyes during night time, an example of this is shown in the figure \ref{fig:new_cw_dark}.

 %Pictures in .eps if you use latex, .pdf or .png if you use pdflatex. Don't specify the extension so you can use both!
\begin{figure}[ht]
  \begin{center}
    \includegraphics*[width=1\textwidth]{new_cw_dark}
  \end{center}
  \caption{A dark style configuration for OpsWeb}
  \label{fig:new_cw_dark}
\end{figure}

\subsection{Live monitoring}
Cluster Web is used for monitoring operations in the control room. When opening the page, it loads a view displaying the schedule for the current day and the days before and after it, however, as time goes by, the page needs to be refreshed to get the current time back to the center of the screen.

In OpsWeb, there is a new feature to help with this, called live mode. When the live mode is enabled by clicking a button in the navigation bar, the schedule moves to the current time so that a configured amount of time is displayed in the past and the future. The schedule timeframe is updated periodically so that the line marking the current time stays fixed on the screen while the schedule moves behind it.

Another feature helping with live monitoring is the countdown clock. It displays how much time there is until the next data item of a data type, for example the next pass. A line is drawn from now line to the next item, and a countdown timer is displayed with it. The format of the countdown can be configured with string placeholders.

 %Pictures in .eps if you use latex, .pdf or .png if you use pdflatex. Don't specify the extension so you can use both!
\begin{figure}[ht]
  \begin{center}
    \includegraphics*[width=1\textwidth]{new_cw_live}
  \end{center}
  \caption{A live monitoring view demonstrating the live mode and the countdown visualization.}
  \label{fig:new_cw_live}
\end{figure}

\subsection{Sharing schedules}
Users can share the exact view they are viewing in OpsWeb with other people by just copying and sending the URL from the browser's address bar, because all the necessary information that defines the view is deep-linked in the URL. For example, the URL can look like this:

http://cluweb.go.esa.int/schedule?configuration=2\&from=2017-10-20T22:00:00\&to=2017-10-21T12:30:00

"From" and "to" parameters define the beginning and the end time of the visible timeframe of the schedule. They are ISO 8601- formatted date strings. "Configuration" parameter is the id of the configuration that is being used.

\subsection{Uberlog integration}
Like Cluster Web, OpsWeb integrates the viewing of Überlog entries, however in a different way. In OpsWeb, the log entries are visible as icons on the timeline, their positions corresponding the log event time. 

Because there are so many entries close to each other, it would not be viable to display them all as the icons would overlap each other when zoomed out. To improve the readability of the timeline, log entries which are close are grouped together and shown as a single icon with a number next to it. The icon being displayed depends on the highest severity level log entry in that group. The groups change dynamically when zoom level changes. This solution was inspired by PowerHA timeline \cite{ibm_timeline} which groups log entries at different severity levels in a similar way.

When the user hovers the cursor over an icon, the times of the first and last entry in the group are displayed in a tooltip as well as the amount of entries for each severity level. If there is only one entry for the icon, the actual text of the log entry is displayed.

The icon grouping was implemented as a generic class that could be used to group any kind of instant events on the timeline.
\section{Architecture}
OpsWeb uses a client-server architecture consisting of a front-end client and a back-end server. Front-end is the user-facing part of the system and contains the application user interface that can be accessed through web browser. Back-end is a server that takes care of user authentication and data storage and processing functionality of the system. The front-end can also request data from other REST APIs if needed. The overall architecture of the system can be seen in the figure \ref{fig:architecture_diagram}.

 %Pictures in .eps if you use latex, .pdf or .png if you use pdflatex. Don't specify the extension so you can use both!
\begin{figure}[ht]
  \begin{center}
    \includegraphics*[width=1\textwidth]{Architecture}
  \end{center}
  \caption{Architecture of OpsWeb}
  \label{fig:architecture_diagram}
\end{figure}

\subsection{Front-end}
The front-end is a web application built using the Angular framework. \cite{angular} Angular provides a clear structure for the different views and services of the application.

Angular applications consist of modules, which are blocks of functionality. In OpsWeb's case, there are three modules; AppModule, ScheduleModule and ImporterModule. AppModule is the root of the application, while ScheduleModule contains functionality specific to the schedule view, and the ImporterModule contains the file importer functionality.

A module can have one or more components, which control the UI views of the application. Components contain the application logic and define the view with HTML- based templates. The templates can contain bindings to the variables of the component class, which allows the view to change automatically when a variable value changes in the component.

Modules also provide services; these are classes which can be used by components to fetch data or run actions in the background. This avoids the problem of duplicating code in case many components need to have access to same functionality, and keeps the application state consistent across different components. In OpsWeb there are services for accessing data from the API and for keeping track of the schedule timeframe.

The schedule visualization itself that is embedded within the schedule component uses D3.js library for drawing the timeline elements. D3.js is a data-driven document library that allows binding data to DOM elements. The schedule is an SVG container in which elements are placed with properties like X and Y coordinates, height and width. With D3.js, elements can be bound to data objects. The objects can have variables such as time\_start and time\_end which then are used to determine both the X position and the width of the element. The Y position of an element depends on which timeline it is placed on; this is determined by the domain variable of the data object, such as "spacecraft".

The most important class that handles the drawing and updating of the schedule is called Gantt. It draws the axes on the schedule and intializes the timeline domains based on the configuration. It also creates classes for each data type defined in the configuration, called "groups".

There are three main group classes handling the actual drawing process of elements: InstantGroup, IntervalGroup and SeriesGroup, handling their respective categories of data which are mentioned in the section \ref{configurability_req}. Each datatype such as passes or SSR fill level is an instance of these classes. They request data from the back-end and use D3.js functions to draw elements on the SVG. When zooming or panning the schedule, functions in these classes are called to update the positions and sizes of the elements accordingly.

New data is requested from the back-end every time the schedule timeframe is moved beyond a certain treshold. The request always has some padding on both sides of the visible timeframe so that small movements of the timeframe don't have to trigger a new request.

\cite{d3js}
\subsection{Back-end}
OpsWeb's backend provides an interface through which the front-end client can receive and send data. It is built using Django REST Framework \cite{django}, which is a Python toolkit for programming web APIs.

OpsWeb's REST API has endpoints for different types of data that the system can visualize, such as passes, bookings and eclipses. The client makes HTTP requests to these endpoints with parameters specifying data filters such as the time range for which data is wanted. The data on back-end is stored in a MariaDB database \cite{mariadb}, which is a fork of MySQL. Django takes care of interfacing with the database with its model system; data models are defined, and data is stored and queried entirely with Python code. The data is then serialized into JSON format and returned in a HTTP response to the client.

The data comes from multiple sources; these are listed in the table \ref{tab:opsweb_data_sources}.

Estrack Management System (EMS) contains information about bookings and ground station visibilities. Importing from EMS is triggered by clicking a button in the client, after which the back-end makes a request to the API, parses the response data and saves it to the database.

Many different types of files need to be manually imported by uploading them through the client; Observation Request (OBRQ) files are provided by the science observation team for each planning period, Short-Term Event Files (STEF) are generated by the flight dynamics team and contain different orbital events which occur during a specified timeframe, and Long-Term Prediction (LTP) files also generated by the flight dynamics team contain information about eclipse times.

SSR fill level works differently from other types of data. It is a synthetic parameter that comes from ESA's MUSTLink (Mission Utility and Support Tools) telemetry database. The fill level values are only valid during passes; between passes, there is no telemetry so the value is at zero even though there is data being recorded. The back-end filters the data so that the invalid values are ignored, and additionally predicts the SSR fill level in the future and between passes based on the TDA modes' data acquisition bitrate.

\begin{table}[!ht]
% Add some padding to the table cells:
\def\arraystretch{1.1}%
\begin{center}
  \caption{OpsWeb data sources}
  \label{tab:opsweb_data_sources}
  \begin{tabular}{| l | l | l | }
    \hline
    Data type & Import method & Source \\
    \hline
    Passes & Triggered from the client & EMS Database  \\
    Bookings & Triggered from the client & EMS Database  \\
    Visibilities & Triggered from the client & EMS Database  \\
    TDA modes & Uploaded from the client & OBRQ files  \\
    Orbital events & Uploaded from the client & STEF files \\
    Eclipses & Uploaded from the client & LTP files \\
    SSR fill level & Not imported; filtered from another API & MUSTLink Database \\
    \hline
  \end{tabular}

  \end{center}
\end{table}

The back-end uses JSON Web Token authentication \cite{jwt}. The client sends a request with the username and password, and receives a token as a response. The token is then placed in the Authorization header in subsequent request to authorize the client. The tokens is set to expire in 5 days unless it is refreshed. Token refresh is done automatically each time page is loaded.

Timeline and style configurations of the front-end client are saved in the back-end's database. Each configuration is a combination of a timeline and a style configuration; these are in separate database tables, but there is a third table that contains the combinations by referring to the timeline and style configuration tables.

\section{Development}
OpsWeb is developed by a team of seven people, including the author of this thesis. The project is led by Cluster II spacecraft operations manager. The author of this thesis worked on this project for six months during an internship. Upon arrival in May 2017, the development had been going on for a couple of months and the main technologies had been chosen. Other than the author during his internship, no one works on the project full-time; instead, people contribute whenever they have time.

Development roles are varied between people depending on their interests and experience, some focusing more on front-end and others on the back-end. 

The author implemented many features mainly on the front-end, namely the live mode, activity countdown, deep linking of the schedule timeframe, visualization of the Uberlog entries, login and authentication, navigation menu and an improved importer view. In addition, miscelllaneous improvements and bug fixes were done and the continuous integration was set up. 

Large features implemented by other developers during this time include configurability of the timeline visualization, refreshing of data when scrolling the timeline, automated testing of the front-end code and major refactoring and improvement work done on the back-end.

In order to keep the front-end code written by several people consistent, linting in the editor is applied, highlighting parts which deviate from the TypeScript guidelines. Similarly, linting could also be applied for the Python code of the back-end.

By the end of the internship in October 2017, OpsWeb had most of the functionality needed for schedule monitoring purposes, which is the first release that will be used by the Cluster flight control team.
\subsection{Scrum}
OpsWeb development team uses Scrum for organizing the project. \cite{beck2001manifesto, scrum} It is an Agile framework within which teams can address the development of complex products. Scrum is intended to be easy to understand and implement in practice. It sets certain events, rules and roles within which the development of a product can happen in a clearly defined way while leaving room for changes.

In Scrum, the project requirements are defined as user stories, which describe some functionality that the product should have from a user's point of view instead of giving a specification for the technical implementation. Product owner is the person who is responsible for providing this overall direction for the project, in OpsWeb's case the product owner is the Cluster II spacecraft operations manager.

The development of OpsWeb is done in two-week sprints. At the beginning of each sprint, the development team allocates user stories between themselves and discusses the technical implementation in a sprint planning meeting. The user stories are split into tasks which need to be completed to implement the feature described in the user story. 

Each user story is scored based on three different categories: concept, front-end and back-end, depending on how much effort the development team thinks the user story will require. "Planning poker" is used for scoring the user stories; each member of the development team votes with a number from Fibonacci or a similar sequence. People who vote with numbers either above or below the average can voice their opinion on why they think the user story would be easier or more difficult than others think. \cite{grenning2002planning}

OpsWeb development team has a daily stand-up meeting every morning to discuss what they have been doing and what they will be doing next, to keep everyone on track about the project.

At the end of each sprint, there is a sprint review meeting in which the current state of the project is shown to the product owner, who can then make requests about what could be improved and what should be focused on during the next sprint.

The Scrum process is managed by a Scrum Master, who makes sure that Scrum is being correctly used for the project. Scrum Master is a member of the development team. For OpsWeb's development, the team uses Taiga web application \cite{taiga} that provides an user interface for managing task status and allocation.

\subsection{Continuous integration}
OpsWeb's code is managed in two GitLab repositories, one for front-end and another for back-end. Initially both used to be in one repository but they were separated for easier and cleaner management.

Both repositories have a master branch that is supposed to contain stable release code, and a development branch that contains new features which have been added after every sprint. User stories are developed separately in their own branches which are based on development branch, and merged back into development branch when ready.

GitLab has built-in continuous integration features which are utilized in OpsWeb's repositories. It's possible to define scripts defining different jobs which are run automatically after a push on a GitLab runner that is a Docker container running on the GitLab server. The idea is to run test first and then deploy the new version on a server.

GitLab's continuous integration jobs are separated into stages which are run in a sequence. In OpsWeb's case what needs to be achieved is to run tests first to validate the commits which were pushed, and if the tests were successful, deploy the new version to a server. Specific jobs can be set to run only for specific branches, which would for example allow deployment to branch-specific servers.

Front-end tests are written using Jasmine framework, and run using Karma test runner on PhantomJS headless browser. The tests define the expected output from functions when given certain parameters as input. Karma runs the tests and then lists which tests succeeded and which failed.

Deployment is done after the test stage has passed. For the front-end this is done by connecting to the server using SSH, pulling the commit there, and then compiling the Typescript. Then the compiled Javascript is copied into a static file folders served by Nginx.

Back-end deployment similarly connects to the server with SSH, pulls the commit and then runs a script to migrate any model changes to the database. Django automatically restarts the server when changes are detected.
\section{Future development}
OpsWeb's first release in late 2017 provides the Cluster II spacecraft operations engineers and spacecraft controllers the ability to display configurable operations schedules. However, it does not yet contain one of the most important features of Cluster Web: being able to schedule passes and export the schedule for mission planning system. This is a feature that users have said to be very important.

Dynamically dragging and dropping elements on the timeline will be a feature to be developed for the next release. Users should be able to add any kind of data to the system in a flexible and intuitive way, whether it is things like spacecraft controller shifts or planned activities for the spacecraft.

The back-end is also for the time being specific to the Cluster mission's use case. Things like SSR fill level prediction do not apply directly to other mission, although similar concepts may be needed. There is also a need for flexible data models; users should be able to define new data types and what fields they contain, and add data to the system directly. Currently the data models are hard-coded in Django.

A new, modular back-end will be built that has a flexible NoSQL  database and a generic data provider interface in which external databases can be plugged in. Mission- specific things like file importers and data processing will be moved into Python modules which can hook into the core system.

As other missions and other users adopt OpsWeb, the development will also eventually open up and allow third parties to do their own additions to the system as needed. If possible, OpsWeb could also be opened to users outside of ESA and spacecraft operations, to allow its use as a fully generic timeline visualization software.

Other tools are also being integrated to the mission control systems, such as BOSS Dashboard. \cite{boss} It provides a modern web-based interface for browsing and visualizing telemetry data. Together with OpsWeb, BOSS Dashboard could be displayed on a big screen in mission control room to allow spacecraft controllers to overview the mission status easily.