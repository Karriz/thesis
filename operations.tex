% Spacecraft operations

This section gives an overview of the day-to-day spacecraft operations, specifically how Cluster Web fits in the big picture to give some context to its usage.

\section{Cluster II}
Cluster II is a mission studying Earth's magnetosphere and its interaction with solar wind. It consists of four spacecraft which orbit the Earth together in a tetrahedral formation. Their orbits have apogees (highest point) at about 120 000 km above the Earth and perigees (lowest point) at about 20 000 km. The eccentric orbits take the spacecraft through some key areas of scientific interest in the magnetosphere, such as the magnetopause and the bow shock. The formation between the four spacecraft is important for determining how the properties of the magnetosphere change over time and location.

Launched in 2000, the mission was originally planned to run until the end of 2003 but it has been extended several times, as of 2017 the mission will run until the end of 2018. While the spacecraft are aging and some systems are no longer operational, they're still producing useful science data.

\section{Mission planning}
The four Cluster II spacecraft are operated from the European Space Operations Centre (ESOC) in Darmstadt, Germany. The scientific operations are planned by Joint Science Operations Team (JSOC) in Chilton, UK.

JSOC sends an OBRQ (Observation Request) file to the flight control team at ESOC for each weekly planning period (PP). The file includes information about which science instruments should be running on the spacecraft at a given time, and what telemetry and data acquisition (TDA) mode should be used. This is planned based on what region of the magnetosphere the spacecraft is flying through.

Mission planner at ESOC uses Cluster Web to create a pass plan for the planning period. Passes are the times when the spacecraft is in contact with a ground station during which data is dumped from the on-board solid state recorder (SSR) and commands can be sent to the spacecraft. 

Mission planner makes sure that the spacecraft is able to contact a ground station during the passes and has sufficient link budget to downlink data. There are different factors affecting the link budget like the spacecraft's distance and elevation angle, and what TDA mode is being used. For maximum efficiency of data collection it is also important to plan the data dump so that the SSR doesn't fill up or empty completely during a pass, but is dumped at sufficient intervals.

After the pass plan (PlanCWEB) file has been created and exported from Cluster Web, it is imported to mission planning system (MPS) together with OBRQ and other relevant files. MPS generates three different types of files: DSF's, SOR's and EVFM's. DSF'S are command sequences to be linked to the spacecraft  by the spacecraft controllers, SOR's are procedures to be executed by the automation system, and EVFM's are jobs to be executed by the ground stations.

The forementioned files are then transferred to servers from which they will be used for controlling the spacecraft and the ground stations to achieve the planned sequence of passes during the next planning period.

The data collected during the passes is tranferred to various different data centres around the world in the distributed Cluster Science Data System (CSDS).

\section{Cluster Web}
Cluster Web is a web portal that the mission planner uses for planning passes as mentioned in the last section. In this section the features and the technology of the software will be detailed more.

The main interface of Cluster Web is a timeline that offers an overview of the mission plan. The timeline is split vertically into four sections, one for each spacecraft in the mission. Different types of data are visualized on the timeline, some of which events or states with a start and end time, some are events at a specific point in time and some are time-value sets shown as graphs on the timeline.

The following table lists the types of visualized data:

\begin{table}[!ht]
% Add some padding to the table cells:
\def\arraystretch{1.1}%
\begin{center}
  \caption{Types of data visualized in Cluster Web}
  \label{tab:clusterweb_data_table}
  \begin{tabular}{| l | l | l | l | l | }
    \hline
    Name & Visualization & Editable & Type & Source \\
    \hline
    Passes & Rectangles & Yes & Operational & Flight control \\
    Visibilities & Rectangles & No & Environment & Flight dynamics \\
    Bookings & Rectangles & No & Operational & Ground stations \\
    TDA modes & Background & No & Operational & Science operations \\
    SSR fill level & Line graph & Indirectly & Resource & Telemetry/prediction \\
    Payload sequences & Rectangles & No & Operational & Science operations \\
    Mneuver times & Diamonds & No & Operational & Flight dynamics \\
    Apogees/perigees & Small points & No & Environment & Flight dynamics \\
    Eclipses & Background & No & Environment & Flight dynamics \\
    \hline
  \end{tabular}

  \end{center}
\end{table}

The data is aggregated from multiple different sources which are imported into Cluster Web's database periodically to keep it up to date. Ground station passes are the only thing that can be edited by the flight control team, rest of the data can either operational events dictated by other teams, naturally occuring events which depend on the orbit of the spacecraft, or resources on board the spacecraft which are influenced by the operational events.

On the user interface, there are buttons for enabling and disabling certain visualizations like SSR fill leve, ground station visibilities (GS), payload sequences (PL) and multiple spacecraft per aperture markings for passes (MSPA). Ground station visibilities and payload sequences cannot be viewed at the same time as they would be on top of each other.

%Pictures in .eps if you use latex, .pdf or .png if you use pdflatex. Don't specify the extension so you can use both!
\begin{figure}[ht]
  \begin{center}
    \includegraphics*[width=1\textwidth]{clusterweb_visibilities}
  \end{center}
  \caption{A screenshot of the Cluster Web timeline interface with ground station visibilities}
  \label{fig:clusterweb_visibilities}
\end{figure}

\subsection{Passes}
Figure~\ref{fig:clusterweb_visibilities} shows the main interface of Cluster Web with ground station visibilities on. The ground station visibilities are displayed as colored bars, each ground station being in its own "lane". Bookings of other missions are shown on top of the visibilities as smaller bars in slightly darker color. From this visualization, the mission planner can determine the times when a ground station is available so that a pass can be scheduled. 

After a pass is created it shows up on the timeline as a rectangle. Different types of passes are displayed in different colors. Light blue rectangle is a normal pass that has both uplink and downlink capacity (commands can be sent to the spacecraft), while a gray pass is downlink-only. If there is a conflict with visibility or another booking, the pass is shown in red color. If the conflict is with another Cluster spacecraft the pass is shown in yellow color.

By clicking a pass, another view opens up that displays further information about the pass. For a pass that was already executed this view displays different kinds of logging data, while for a pass that is in the future, the view shows a link budget graph that can be adjusted by changing parameters.

\subsection{SSR fill level}
Another thing that can be seen in Figure~\ref{fig:clusterweb_visibilities} is the SSR fill level graph. It is shown for each spacecraft as a red and green line, in which the red part is drawn based on telemetry data about the fill level (that is available until the end of latest pass), and the green part is based on prediction. 

The prediction model works by taking into account the data accumulation bitrates based on TDA modes and downlink bitrates during passes. When passes are edited this also changes the SSR fill level prediction. 

It is important to schedule the passes so that the fill level doesn't completely empty during a pass as that would be a waste of ground station time, or completely fill between passes because then the spacecraft couldn't collect more data before the memory is emptied. Because of the prediction visualization that changes depending on the passes the mission planner can easily do scheduling while taking the SSR fill level into account.
