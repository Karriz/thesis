% Spacecraft operations

This section gives an overview of the day-to-day spacecraft operations, specifically how Cluster Web fits in the big picture to give some context to its usage.

\section{Cluster II}
Cluster II is a mission studying Earth's magnetosphere and its interaction with solar wind. It consists of four spacecraft which orbit the Earth together in a tetrahedral formation. Their orbits have apogees (highest point) at about 120 000 km above the Earth and perigees (lowest point) at about 20 000 km. The eccentric orbits take the spacecraft through some key areas of scientific interest in the magnetosphere, such as the magnetopause and the bow shock. The formation between the four spacecraft is important for determining how the properties of the magnetosphere change over time and location.

Launched in 2000, the mission was originally planned to run until the end of 2003 but it has been extended several times, as of 2017 the mission will run until the end of 2018. While the spacecraft are aging and some systems are no longer operational, they're still producing useful science data.

\section{Mission planning}
The four Cluster II spacecraft are operated from the European Space Operations Centre (ESOC) in Darmstadt, Germany. The scientific operations are planned by Joint Science Operations Team (JSOC) in Chilton, UK.

JSOC sends an OBRQ (Observation Request) file to the flight control team at ESOC for each weekly planning period (PP). The file includes information about which science instruments should be running on the spacecraft at a given time, and what telemetry and data acquisition (TDA) mode should be used. TDA mode affects what telemetry the spacecraft is transmitting and at what rate it saves data to the on-board solid state recorder (SSR).

Mission planner at ESOC uses Cluster Web to create a pass plan for the planning period. Passes are the times when the spacecraft is in contact with a ground station during which the ground station can uplink commands to the spacecraft and listen to the downlink signal from the spacecraft that includes telemetry and usually science data that is being dumped from the solid state recorder.

When scheduling passes, mission planner makes sure that the spacecraft signal can be heard from the ground station during the pass and has a sufficient link budget to downlink data at the planned bitrate. There are different factors affecting the link budget like the spacecraft's distance and elevation angle, and what TDA mode is being used. For maximum efficiency it is also important to plan the data dump times so that the SSR doesn't get full between passes or empty completely during a pass.

After the pass plan (PlanCWEB) file has been created and exported from Cluster Web, it is imported to mission planning system (MPS) together with OBRQ and other relevant files. MPS generates three different types of files: DSF's, SOR's and EVFM's. DSF'S are command sequences to be linked to the spacecraft  by the spacecraft controllers, SOR's are procedures to be executed by the automation system, and EVFM's are jobs to be executed by the ground stations.

The forementioned files are then transferred to servers from which they will be used for controlling the spacecraft and the ground stations to achieve the planned sequence of passes during the next planning period.

The data collected during the passes is tranferred to various different data centres around the world in the distributed Cluster Science Data System (CSDS).

\section{Cluster Web}
Cluster Web is a web portal that the mission planner uses for planning passes as mentioned in the last section. In this section the features of the software will be detailed more.

In addition to pass planning, Cluster Web is also used for overviewing the operations schedule by displayign the timeline on a big screen on the flight control room, and during meetings engineers use it to visualize relevant information that's being discussed, like the pass times for the upcoming week.

The main interface of Cluster Web is a timeline that offers an overview of the mission plan. The timeline is split vertically into four sections, one for each spacecraft in the mission. Different types of data are visualized on the timeline, some of which events or states with a start and end time, some are events at a specific point in time and some are time-value sets shown as graphs on the timeline.

Table~\ref{tab:clusterweb_data_table} lists the types of visualized data:

\begin{table}[!ht]
% Add some padding to the table cells:
\def\arraystretch{1.1}%
\begin{center}
  \caption{Types of data visualized in Cluster Web}
  \label{tab:clusterweb_data_table}
  \begin{tabular}{| l | l | l | l | l | }
    \hline
    Name & Visualization & Editable & Type & Source \\
    \hline
    Passes & Rectangles & Yes & Operational & Flight control \\
    Visibilities & Rectangles & No & Environment & Flight dynamics \\
    Bookings & Rectangles & No & Operational & Ground stations \\
    TDA modes & Background & No & Operational & Science operations \\
    SSR fill level & Line graph & Indirectly & Resource & Telemetry/prediction \\
    Payload sequences & Rectangles & No & Operational & Science operations \\
    Maneuver times & Diamonds & No & Operational & Flight dynamics \\
    Apogees/perigees & Small points & No & Environment & Flight dynamics \\
    Eclipses & Rectangles & No & Environment & Flight dynamics \\
    \hline
  \end{tabular}

  \end{center}
\end{table}

The data is aggregated from multiple different sources which are imported into Cluster Web's database periodically to keep it up to date. Ground station passes are the only thing that can be edited by the flight control team, rest of the data is either operational events dictated by other teams, naturally occuring events which depend on the orbit of the spacecraft, or resources on board the spacecraft which are influenced by the operational events.

On the user interface, there are buttons for enabling and disabling certain visualizations like SSR fill leve, ground station visibilities (GS), payload sequences (PL) and multiple spacecraft per aperture markings for passes (MSPA). Ground station visibilities and payload sequences cannot be viewed at the same time as they would be on top of each other.

%Pictures in .eps if you use latex, .pdf or .png if you use pdflatex. Don't specify the extension so you can use both!
\begin{figure}[ht]
  \begin{center}
    \includegraphics*[width=1\textwidth]{clusterweb_visibilities}
  \end{center}
  \caption{A screenshot of the Cluster Web timeline interface with ground station visibilities}
  \label{fig:clusterweb_visibilities}
\end{figure}

\subsection{Pass scheduling}
Figure~\ref{fig:clusterweb_visibilities} shows the main interface of Cluster Web with ground station visibilities on. The ground station visibilities are displayed as colored bars, each ground station being in its own "lane". Bookings of other missions are shown on top of the visibilities as smaller bars in slightly darker color. From this visualization, the mission planner can determine the times when a ground station is available so that a pass can be scheduled. 

After a pass is created it shows up on the timeline as a rectangle. Different types of passes are displayed in different colors. Light blue rectangle is a normal pass that has both uplink and downlink capacity (commands can be sent to the spacecraft), while a gray pass is downlink-only. If there is a conflict with visibility or another booking, the pass is shown in red color. If the conflict is with another Cluster spacecraft the pass is shown in yellow color.

By clicking a pass, another view opens up in which the user can edit the pass properties, and access different kinds of information about the pass such as a bitrate forecast and log entries. There is also a checklist of things which should be checked every time a pass is edited.

%Pictures in .eps if you use latex, .pdf or .png if you use pdflatex. Don't specify the extension so you can use both!
\begin{figure}[ht]
  \begin{center}
    \includegraphics*[width=1\textwidth]{clusterweb_pass_edit}
  \end{center}
  \caption{A screenshot of Cluster Web pass editing and info view}
  \label{fig:clusterweb_pass_edit}
\end{figure}

\subsection{SSR fill level graph}
Another important visualization that can be seen in Figure~\ref{fig:clusterweb_visibilities} is the SSR fill level graph. It is shown for each spacecraft as a red and green line, in which the red part is drawn based on telemetry data about the fill level (that is available until the end of latest pass), and the green part is based on prediction. 

The prediction model takes into account the data recording bitrates based on TDA modes and downlink bitrates during passes. When pass times and modes are edited this also changes the SSR fill level prediction. Visualizing this prediction makes it convenient for the mission planner to see what are good times for dumping the data.

\subsection{Eclipse times}

%Pictures in .eps if you use latex, .pdf or .png if you use pdflatex. Don't specify the extension so you can use both!
\begin{figure}[ht]
  \begin{center}
    \includegraphics*[width=1\textwidth]{clusterweb_eclipses}
  \end{center}
  \caption{A screenshot of the Cluster Web timeline interface showing an eclipse season}
  \label{fig:clusterweb_eclipses}
\end{figure}

Cluster Web displays the times each spacecraft is in the shadow of Earth or the Moon, i.e. in an eclipse. During full eclipses which can last a few of hours the spacecraft doesn't receive any power from solar cells so it shuts down.

Most of the time there are no Earth eclipses, they only happen in two eclipse seasons during a year which can last a week or two. During these seasons the spacecraft go through an eclipse once per orbit, that is about every 54 hours.

Before an eclipse some preparation needs to be done, like dumping the SSR data and shutting down the science instruments. After an eclipse the spacecraft starts receiving power and wakes up in a fail-safe mode. In this mode the spacecraft doesn't transmit telemetry or collect data so a pass is required to set the spacecraft back to operational configuration.

 An eclipse season requires special planning and attention to keep the spacecraft healthy, so it is useful to be able to overview the situation in Cluster Web well in advance.
 
 Eclipses are displayed as black rectangles as can be seen in Figure~\ref{fig:clusterweb_eclipses}, and by hovering mouse over them some additional information can be seen like which engineers are assigned to the operations of that eclipse, and if the spacecraft will power down.
 
 \subsection{Überlog integration}
 Cluster Web is integrated with another tool that is used by spacecraft controllers, named Überlog. It is a web application for logging operations events like opening and closing communications links, and any issues that may occur. Automation system also inserts log entries when it conducts operations by itself.
 
 On Cluster Web, the log entries can be displayed on a sidebar in the schedule view and filtered by spacecraft, or in the pass information view for that particular pass.
 
 This functionality can for example be used in meetings by displaying log entries that occured during a pass when troubleshooting an issue.
 
 %Pictures in .eps if you use latex, .pdf or .png if you use pdflatex. Don't specify the extension so you can use both!
\begin{figure}[ht]
  \begin{center}
    \includegraphics*[width=1\textwidth]{clusterweb_uberlog}
  \end{center}
  \caption{Cluster Web's Überlog sidebar}
  \label{fig:clusterweb_uberlog}
\end{figure}